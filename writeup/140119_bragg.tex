\documentclass[11pt,letter]{article}
\usepackage[top=0.65in,bottom=0.9in,left=0.85in,right=0.85in]{geometry}

%\def\baselinestretch{1.25}
\def\baselinestretch{1.0}

\usepackage[greek, english]{babel}
\usepackage{multicol}

\usepackage[draft]{graphicx}
\usepackage[export]{adjustbox}


% The use of the times package forces the use of the type-1 times
% roman font, but the times roman font does not look nice.
% Besides the times roman font still does not print correctly on
% the dopy printer.
%\usepackage{times}


\usepackage{fancyhdr}
\usepackage{amsmath}
\usepackage{bm}
\usepackage{bbold}
\usepackage{parskip}

\newcommand{\bv}[1]{\ensuremath{\bm{#1}}}
\newcommand{\Lc}{\ensuremath{L_{\mathrm{c}}}}
\newcommand{\dsig}[1]{\ensuremath{ \frac{ d\,\sigma_{#1} }{d\,\Omega} }}

\newcommand{\iisat}{\ensuremath{I_{\mathrm{p}}/I_{\mathrm{sat}}}}
\newcommand{\itof}{\ensuremath{I_{\infty} }}

\begin{document}

\section{Scattering of light by an array of atoms}

In our experiment we observe the scattering of photons from atoms confined in
an optical lattice, here we treat this situation by obtaining the the field
scattered from a single atom and then summing the field contributions from all
the atoms coherently at the location of our detector.  The main goal of this
document is to find the connection between the intensity that we measure in our
cameras and the spin structure factor as it is calculated by the theorists in
our collaboration.  

\subsection{Electric field and intensity due to a single atom}

To calculate the scattered field, one uses the source-field expression, which
relates the radiated field to the emitting dipole moment, this is derived in
the standard textbooks~\cite{loudon2000quantum,cohen1998atom}.  The field at
the position of the detector $\bv{r}_{D}$ is given by
\begin{equation} 
    E^{(+)}( \bv{r}_{D}, t) = 
    \eta e^{- i \omega_{L} ( t -r_{D}/c) } 
    S_{-}\left(t - \frac{ r_{D} }{c} \right)
    \label{eq:source-field} 
\end{equation} where $\eta$
is a proportionality factor that we will address later on.  The time-averaged
intensity at the detector is 
\begin{equation}
\label{eq:Idef}
\begin{split}
\langle I (t) \rangle = & 
    \langle E^{(-)}(\bv{r}_{D}, t) E^{(+)}(\bv{r}_{D}, t) \rangle \\
   = & |\eta|^{2} \langle S_{+}(t-r_{D}/c)S_{-}(t-r_{D}/c) \rangle  \\
   = & |\eta|^{2} \langle S_{+}S_{-} \rangle \\
   = & |\eta|^{2} \rho_{ee}  
\end{split} 
\end{equation}
In the third line the time dependence is dropped since we are interested in the
steady state solution.  We can write the $S_{\pm}$ operators of the atoms as 
\begin{equation} 
    S_{\pm} = \langle S_{\pm} \rangle + \delta S_{\pm} 
\end{equation}
which defines the difference, $\delta S$, between $S_{\pm}$ and its average
value.   Writing $S_{\pm}$ this way allows us to distinguish between two
components in the radiated light,  the radiation of the average dipole $\langle
S_{\pm}\rangle$ which is the radiation of a classical oscillating dipole with a
phase that is well defined relative to the incident laser field, and the
radiation form the $\delta S_{\pm}$ component which does not have a phase that
is well defined relative to the incident field because it comes form the
fluctuating part of the atomic dipole.  The intensity is then a sum of coherent
and incoherent parts 
\begin{equation} 
I  = \eta^{2} \langle S_{+}\rangle \langle S_{-} \rangle 
   + \eta^{2} \langle \delta S_{+} \delta S_{-} \rangle 
\end{equation}
where we have used the fact that by definition $\langle \delta S_{\pm}
\rangle = 0$. The first and second terms of this equation are the coherent
and incoherent intensity which can be calculated by using the
steady-state solutions to the optical Bloch equations given by
\begin{gather} 
    \langle S_{\pm} \rangle =  u \pm i v  \\
    u =  \frac{ \Delta }{ \Gamma  \sqrt{ I_{\mathrm{p}} / I_{\mathrm{sat}}} } 
         \frac{s}{ 1 + s } \\
    v =  \frac{ 1 } { 2 \sqrt{ I_{\mathrm{p}} / I_{\mathrm{sat}}} } 
         \frac{s}{1+s} \\
\end{gather}
where $s$ is the saturation parameter for an incident probe with intensity
$I_{\mathrm{p}}$:
\begin{equation}
s = \frac{ 2  \iisat } { 1 + 4(\Delta/\Gamma)^{2} } 
  =  \frac{ s_{0} } { 1 + 4(\Delta/\Gamma)^{2} }
\end{equation}
and we have defined $s_{0} = 2 \iisat$. 
The coherent and incoherent intensities are
\begin{equation}
\begin{split} 
    \frac{1}{\eta^{2}}  I_{\mathrm{coh}} = &
        \frac{1}{2} \frac{s}{(1+s)^{2} } 
      = \rho_{ee}  \frac{1}{1+s} \\
    \frac{1}{\eta^{2}}  I_{\mathrm{incoh}}  = & 
        \langle S_{+}S_{-} \rangle - \langle S_{+} \rangle \langle S_{-} \rangle
      = \frac{1}{2} \frac{s^{2}}{(1+s)^{2}} = \rho_{ee} \frac{s}{1+s}
 \label{eq:coh-incoh} 
\end{split}
\end{equation}
Note that if we add up coherent and incoherent part we get the more familiar
result $I=\eta^{2} \rho_{ee}$, where the total intensity is simply proportional
to the population of the excited state.


\subsection{Scattering cross-section}

Now we will turn onto the evaluation of $\eta$, the proportionality factor
between the field and the emitting dipole.  Knowledge of $\eta$ will allow us
to sum coherently the field from a collection of atoms. 

We start by considering the  transition matrix element between the following
initial and final states of the atom+photon system: 
\begin{equation}
\begin{split}
    | \varphi_{i} \rangle = & | g ; \bv{k}\bv{\varepsilon} \rangle \\
    | \varphi_{f} \rangle = & | g ; \bv{k}'\bv{\varepsilon}' \rangle \\
\end{split}
\end{equation}
These states represent the absorption and re-emission of a single photon by the
atom, with two possibly different initial and final photon states.
 
The transition rate to from $i\rightarrow f$ is given by 
\begin{equation}
    \label{eq:transitionRate}
    w_{fi} = \frac{2\pi}{\hbar} | \mathcal{T}_{fi} |^{2} \delta(E_{f}-E_{i})
\end{equation}
Where we use the notation in~\cite{cohen1998atom} (see Exercise~5 on pg. 530),
and $\mathcal{T}_{fi}$ is given by
\begin{equation}
    \mathcal{T}_{fi} = \frac{  
        \langle g; \bv{k}'\bv{\varepsilon}'| H_{I}' | e; 0 \rangle 
        \langle e; 0 | H_{I}' | g; \bv{k}\bv{\varepsilon} \rangle }
        { \hbar\omega - \hbar\omega_{0} + i\hbar (\Gamma/2 ) }
\end{equation} 
where $H_{I}'$ is the interaction Hamiltonian
\begin{equation}
    H_{I}' = -\bv{d} \cdot \bv{E}_{\perp} ( \bv{r} ) 
\end{equation}
and\footnote{Notice that the presence of $\epsilon_{0}$ reveals that we are
using SI units, following the treatment in~\cite{cohen1998atom}.}
\begin{equation}
    \bv{E}_{\perp}(\bv{r}) = i \sum_{j} 
        \left[ \frac{ \hbar \omega_{j} }{ 2\varepsilon_{0} L^{3}}  \right]^{1/2}
        \left( \hat{a}_{j}\bv{\varepsilon}_{j} e^{i\bv{k}_{j}\cdot\bv{r}} 
              - \hat{a}_{j}^{+}\bv{\varepsilon}_{j}^{*} e^{-i\bv{k}_{j}
                \cdot\bv{r}} 
        \right)
\end{equation}
Notice that there is an intermediate excited state $|e;0\rangle$, since the
absorption-emission event is a second order process.

Using the expressions for $H_{I}'$ and $\bv{E}_{\perp}(\bv{r})$ we obtain for
the matrix element 
\begin{equation}
   \langle e; 0 | H_{I}' | g; \bv{k}\bv{\varepsilon} \rangle = 
       -i \sqrt{ \frac{ \hbar \omega }{2 \varepsilon_{0} L^{3} }} 
      \langle e | (\bv{d} \cdot \bv{\varepsilon}^{*} ) 
       e^{-i\bv{k}\cdot\bv{r}}| g \rangle
\end{equation}
At this point the textbook treatment usually assumes that the atom is at the
origin and that the size of the atom wavefunction is very small compared to
$|\bv{k}|^{-1}$, and so the exponential inside the matrix element typically
does not show up.  In our case the atom is in a lattice site and its center of
mass state is one of the harmonic oscillator states of a lattice well is large
enough that the exponential term cannot be neglected.  
 
The states $|e\rangle$ and $|g\rangle$ include the center of mass and internal
states of the atom.  We separate the center of mass part, and keep the labels
$e$, $g$ for the internal states. Also, we denote the center of mass initial
and final states as $| u \rangle$ and $|u'\rangle$ respectively, and the center
of mass state of the intermediate excited state as $| v\rangle$, we have
\begin{equation}
   \langle e; 0 | H_{I}' | g; \bv{k}\bv{\varepsilon} \rangle = 
       -i \sqrt{ \frac{ \hbar \omega }{2 \varepsilon_{0} L^{3} }} 
      \langle e | \bv{d} \cdot \bv{\varepsilon}^{*} | g \rangle 
      \langle v | e^{-i\bv{k}\cdot\bv{r}} | u \rangle
\end{equation}
and similarly
\begin{equation}
   \langle g; \bv{k}'\bv{\varepsilon}' | H_{I}' | e; 0\rangle = 
       i \sqrt{ \frac{ \hbar \omega' }{2 \varepsilon_{0} L^{3} }} 
      \langle g | \bv{d} \cdot \bv{\varepsilon}' | e \rangle 
      \langle u' | e^{i\bv{k}'\cdot\bv{r}} | v \rangle
\end{equation}
This gives for the matrix element
\begin{equation}
    \mathcal{T}_{fi} = \sum_{v} \frac{\sqrt{\omega \omega'}}
                                     {2\varepsilon_{0} L^{3}}
    \frac{ 
      \langle g | \bv{d} \cdot \bv{\varepsilon}' | e \rangle 
      \langle e | \bv{d} \cdot \bv{\varepsilon}^{*} | g \rangle 
      \langle u'| e^{i\bv{k}'\cdot\bv{r}} | v \rangle 
      \langle v | e^{-i\bv{k}\cdot\bv{r}} | u  \rangle
       }
        { \omega - \omega_{0} + i (\Gamma/2 ) }
\end{equation}
where we have summed over all possible intermediate center of mass states.
Note that the sum can be taken out using the closure relation
$\sum_{v}|v\rangle\langle v| = \mathbb{1}$. 
%We note here that the phase factors $i\bv{k}\cdot\bv{r}$ that appear due to
%the center of mass motion of the atom are uncorrelated.   This comes out that
%way because the scattering of a photon is second order: first the photon gets
%absorbed and then it gets reemitted at a later time.   In neutron scattering
%for example, the scattering process is first order, represented by a contact
%interaction between the neutron and the nuclei in the crystal,  In this case
%the phase factor due to the center of mass motion shows up as \begin{equation}
%\langle  e^{i (\bv{k}'-\bv{k}) \cdot \bv{r}} \rangle_{\mathrm{CM}}
%\end{equation} The norm squared of this term is the Debye-Waller factor.   It
%is necessary to consider if the uncorrelated expectation values that appear in
%the second order photon scattering are correct.  To this effect one needs to
%consider the time between absorption and emission processes which is on the
%order of $\Gamma$, the linewidth of the excited state.   This is much larger
%that the typical harmonic oscillator frequency in a lattice site, which is
%about 400 kHz for a lithium atom in a 50 recoil lattice.   With this in mind
%we may think about photon scattering as an effectively first order process
%(scattering and emission happen much quicker than the atom's center of mass
%can move)  and we may take the liberty of writing the transition matrix
%element which includes the center of mass motion as \begin{equation}
%\begin{split} \langle \mathcal{T}_{fi} \rangle_{\mathrm{CM}} = & \left \langle
%\frac{  \langle g; \bv{k}'\bv{\varepsilon}'| H_{I}' | e; 0 \rangle \langle b;
%0 | H_{I}' | g; \bv{k}\bv{\varepsilon} \rangle } { \hbar\omega -
%\hbar\omega_{0} + i\hbar (\Gamma/2 ) } \right \rangle _{\mathrm{CM}} \\ =&
%\frac{\sqrt{\omega \omega'}}{2\varepsilon_{0} L^{3}} \frac{ \langle g | \bv{d}
%\cdot \bv{\varepsilon}' | e \rangle \langle e | \bv{d} \cdot \bv{\varepsilon}
%| g \rangle \langle  e^{i(\bv{k}'-\bv{k}) \cdot\bv{r}} \rangle_{\mathrm{CM}} }
%{ \omega - \omega_{0} + i (\Gamma/2 ) } \end{split} \end{equation} where the
%more famliar Debye-Waller factor shows up.  

In our experiment we are driving a sigma-minus transition so we can consider
only the projection of $\bv{d}$ onto $\bv{\varepsilon}_{-}$ 
\begin{equation}
     \langle e | \bv{d} \cdot \bv{\varepsilon}^{*} | g \rangle  \equiv
     d_{-} (\bv{\varepsilon}_{-}  \cdot \bv{\varepsilon}^{*} )
\end{equation} 
which leads to 
\begin{equation}
    \mathcal{T}_{fi}  = 
    \frac{\sqrt{\omega \omega'}}{2\varepsilon_{0} L^{3}}
    \frac{ |d_{-}|^{2}  (\bv{\varepsilon}_{+}\cdot \bv{\varepsilon}' )
                       (\bv{\varepsilon}^{*}\cdot \bv{\varepsilon}_{-} )}
        { \omega - \omega_{0} + i (\Gamma/2 ) }
      \langle u' | e^{i(\bv{k}'-\bv{k}) \cdot\bv{r}} | u  \rangle
\end{equation}
We use the relation between $|d_{-}|^{2}$ and the linewidth of the transition
\begin{equation} 
    |d_{-}|^{2} =  3\pi \varepsilon_{0} \hbar
  \left( \frac{c}{\omega_{0}} \right)^{3}  \Gamma
\end{equation}
and also the approximation $\omega' \approx \omega \approx \omega_{0}$ for the
square root in the denominator to obtain
\begin{equation}
    \mathcal{T}_{fi} =
    \frac{ 3 } {k^{2}} 
    \frac{ \pi \hbar c } {  L^{3} } 
        (\bv{\varepsilon}_{+}\cdot \bv{\varepsilon}' )
                       (\bv{\varepsilon}^{*}\cdot \bv{\varepsilon}_{-} )
    \frac{ \Gamma/2  }
        { \omega - \omega_{0} + i (\Gamma/2 ) }
      \langle u' | e^{i(\bv{k}'-\bv{k}) \cdot\bv{r}} | u  \rangle
\end{equation}

To obtain the scattering rate of photons towards a certain solid angle
$\Omega'$, we must sum over all values of $k'$ in the direction of $\Omega'$.
The number of final states with energy between $\hbar c k'$ and $\hbar c ( k' +
\mathrm{d}k')$  whose wave vector points inside the solid angle $\mathrm{d}
\Omega'$ equals 
\begin{equation}
    \rho( \hbar c k') \hbar c \mathrm{d} k' \mathrm{d} \Omega ' 
  = \frac{L^{3}}{8 \pi^{3} }  k'^{2} \mathrm{d} k' \mathrm{d} \Omega' 
\end{equation}
where $\rho$ is the density of states, which is a function of the photon energy
$\hbar c k$.  We use the density of states to replace the sum over $k'$ with an
integral,  and obtain the total transition rate in the direction~$\Omega'$: 
\begin{equation}
\begin{split}
  \sum_{f} w_{fi} = & 
   \frac{2\pi}{\hbar}  \mathrm{d} \Omega' 
      \int_{0}^{\infty} \frac{k'^{2} \mathrm{d} k' }{ (2\pi / L^{3} ) ^{3} } 
   | \mathcal{T}_{fi} |^{2} 
   \delta( \hbar c k' - \hbar c k )  \\ 
   = & 
   \mathrm{d} \Omega' \frac{9}{4 k^{2}} \frac{ c } {L^{3} }
        |(\bv{\varepsilon}_{+}\cdot \bv{\varepsilon}' )
                       (\bv{\varepsilon}^{*}\cdot \bv{\varepsilon}_{-} ) |^{2}
    \left|
    \frac{ \Gamma/2  }
        { \omega - \omega_{0} + i (\Gamma/2 ) }  
      \langle u' | e^{i(\bv{k}'-\bv{k}) \cdot\bv{r}} | u  \rangle
     \right| ^{2} \\ 
   = & 
   \mathrm{d} \Omega' \frac{9}{4 k^{2}} \frac{ c } {L^{3} }
        |(\bv{\varepsilon}_{+}\cdot \bv{\varepsilon}' )
                       (\bv{\varepsilon}^{*}\cdot \bv{\varepsilon}_{-} ) |^{2}
    \frac{ (\Gamma/2)^{2}  }
        { \Delta^{2} +  (\Gamma/2 )^{2} }
    \left|
      \langle u' | e^{i(\bv{k}'-\bv{k}) \cdot\bv{r}} | u  \rangle
\right| ^{2} \\ 
\end{split} 
\end{equation}
If we consider the flux corresponding to the state of the initial photon $\phi
= c/L^{3}$ then we can define the differential cross section 
\begin{equation}
 \frac{ \mathrm{d} \sigma } { \mathrm{d} \Omega'} =  
    \frac{\sum_{f} w_{fi} } { \mathrm{d} \Omega' \phi} = 
    \frac{9}{4 k^{2}} 
        |(\bv{\varepsilon}_{+}\cdot \bv{\varepsilon}' )
                       (\bv{\varepsilon}^{*}\cdot \bv{\varepsilon}_{-} ) |^{2}
    \frac{ (\Gamma/2)^{2}  }
        { \Delta^{2} +  (\Gamma/2 )^{2} }
    \left|
      \langle u' | e^{i(\bv{k}'-\bv{k}) \cdot\bv{r}} | u  \rangle
\right| ^{2}  
\end{equation}
From here we can write down the intensity on a detector located at $\bv{r}_{D}$
in the direction of $\mathrm{d} \Omega'$ as \footnote{Later on we will sum over
output polarizations and final center of mass states of the atom, since our
detection is insensitive to them.}
\begin{equation}
\begin{split}
I  =& \frac{1}{r_{D}^{2}} \frac{ \mathrm{d} \sigma } { \mathrm{d} \Omega'}
      I_{\mathrm{p}} 
   =  \frac{1}{r_{D}^{2}} \frac{ \mathrm{d} \sigma } { \mathrm{d} \Omega'}
      \frac{\hbar c k^{3}\Gamma}{6 \pi} 
      \frac{ I_{\mathrm{p}}}{I_{\mathrm{sat}}}  \\ 
   =& \frac{\hbar c k \Gamma}{r_{D}^{2}}  
    \frac{9}{4 (6\pi)} 
        |(\bv{\varepsilon}_{+}\cdot \bv{\varepsilon}' )
                       (\bv{\varepsilon}^{*}\cdot \bv{\varepsilon}_{-} ) |^{2}
    \left|
      \langle u' | e^{i(\bv{k}'-\bv{k}) \cdot\bv{r}} | u  \rangle
  \right| ^{2}
     \frac{ s_{0}/2   }
        { 4(\Delta/\Gamma)^{2} + 1 }
\end{split}
\end{equation}
We identify the last fraction in this product as $\rho_{ee}$ (in the limit of
low intensity). Comparing with Eq.~\ref{eq:coh-incoh} we can write down an
expression for $\eta$, 
\begin{equation}
  \eta = \left[ \frac{\hbar c k \Gamma}{r_{D}^{2}}  
    \frac{9}{24\pi} \right]^{1/2} 
        (\bv{\varepsilon}_{+}\cdot \bv{\varepsilon}' )
                       (\bv{\varepsilon}^{*}\cdot \bv{\varepsilon}_{-} ) 
      \langle u' | e^{i(\bv{k}'-\bv{k}) \cdot\bv{r}} | u  \rangle
\end{equation}
With an exact expression for $\eta$ we can obtain the field radiated by each
atom and proceed to sum the field coherently for a collection of atoms. 

\subsection{Summation for a collection of atoms} 

For a collection of atoms, the resulting field is the sum of the field produced
by each individual atom, so we have  
\begin{equation}
\begin{split}
\langle I (t) \rangle = & 
    \left\langle \left( \sum_{m} E_{m}^{(-)}(\bv{r}_{D}, t) \right)
            \left( \sum_{n} E_{n}^{(+)}(\bv{r}_{D}, t) \right) \right\rangle \\
\end{split} 
\end{equation}
where we have labeled the atoms with the indices $m$ and $n$.  We insert the
source-field expression from Eq.~\ref{eq:source-field} (dropping the time
dependence) 
\begin{equation}
\begin{split}
 I = &
    \sum_{mn}  \eta_{m}\eta_{n}^{*}  
              \left\langle S_{m+}S_{n-} \right\rangle
\end{split} 
\end{equation}
Using $S=\langle S \rangle + \delta S$, as we did above to obtain the coherent
and incoherent parts of the intensity, we obtain
\begin{equation}
\begin{split}
 I  = &
    \sum_{mn}  \eta_{m}\eta_{n}^{*} \left(
              \langle S_{m+}\rangle \langle S_{n-} \rangle  
            + \langle \delta S_{m+} \delta S_{n-} \rangle \right) \\
    = & 
    \sum_{mn}  \eta_{m}\eta_{n}^{*} 
        \langle  S_{m+}\rangle \langle S_{n-} \rangle 
   + \sum_{n} | \eta_{n}|^{2} \langle \delta S_{n+} \delta S_{n-} \rangle 
\end{split} 
\end{equation}
The steady state solutions of the optical Bloch equations are used again to
evaluate the expectation values and we obtain for $I$
\begin{multline}
 I = 
  \sum_{mn}  \eta_{m}\eta_{n}^{*}
    \left(
    \frac{ \Delta_{m} }{ \Gamma  \sqrt{ I_{\mathrm{p}} / I_{\mathrm{sat}}} } 
    \frac{s_{m}}{ 1 + s_{m} } 
   + i 
    \frac{ 1 } { 2 \sqrt{ I_{\mathrm{p}} / I_{\mathrm{sat}}} } 
    \frac{s_{m}}{1+s_{m}} 
    \right) 
    \left(
    \frac{ \Delta_{n} }{ \Gamma  \sqrt{ I_{\mathrm{p}} / I_{\mathrm{sat}}} } 
    \frac{s_{n}}{ 1 + s_{n} } 
   - i 
    \frac{ 1 } { 2 \sqrt{ I_{\mathrm{p}} / I_{\mathrm{sat}}} } 
    \frac{s_{n}}{1+s_{n}} 
    \right) \\
   + \sum_{n} | \eta_{n}|^{2} \frac{1}{2} \frac{ s_{n}^{2} } 
                                               { (1 + s _{n} )^{2} } 
\end{multline}
\begin{multline}
 I  = 
  \sum_{mn}  \eta_{m}\eta_{n}^{*}
    \frac{ s_{m} s_{n} } { (\iisat) ( 1+s_{m} )( 1+s_{n} ) }
    \left(
        \frac{ \Delta_{m} \Delta_{n} }{ \Gamma^{2} } 
      + i \frac{ \Delta_{n} }{ 2 \Gamma } 
      - i \frac{ \Delta_{m} }{ 2 \Gamma } 
      + \frac{1}{4}  
    \right)  
   + \sum_{n} | \eta_{n}|^{2} \frac{1}{2} \frac{ s_{n}^{2} } 
                                               { (1 + s _{n} )^{2} } 
\end{multline}
The last term here is the incoherently scattered part due to the fluctuating
fraction $\delta S_{\pm}$ of the atomic dipole.  The cross terms do not appear
in this sum because $\langle \delta S_{m+} \delta S_{n-}\rangle =0$ for $m\neq
n$,  this is in fact why this part is identified as the incoherent scattering. 

We proceed to split up the first sum into same-atom ($n=m$) and different atom
($n<m$) parts 
\begin{multline}
 I  = 
  \sum_{m<n} 
    \frac{ s_{m} s_{n} } { (\iisat) ( 1+s_{m} )( 1+s_{n} ) }
    \left(
        \eta_{m}\eta_{n}^{*}
    \left(
        \frac{ \Delta_{m} \Delta_{n} }{ \Gamma^{2} } 
      + i \frac{ \Delta_{n} }{ 2 \Gamma } 
      - i \frac{ \Delta_{m} }{ 2 \Gamma } 
      + \frac{1}{4}  
    \right)  \right. \\
   \left.  + 
        \eta_{n}\eta_{m}^{*}
    \left(
        \frac{ \Delta_{n} \Delta_{m} }{ \Gamma^{2} } 
      + i \frac{ \Delta_{m} }{ 2 \Gamma } 
      - i \frac{ \Delta_{n} }{ 2 \Gamma } 
      + \frac{1}{4}  
    \right) 
    \right)  \\
  + \sum_{n}  |\eta_{n}|^{2}
    \frac{ s_{n} s_{n} } { (\iisat) ( 1+s_{n} )( 1+s_{n} ) }
    \left(
        \frac{ \Delta_{n} \Delta_{n} }{ \Gamma^{2} } 
      + \frac{1}{4}  
    \right) \\ 
   + \sum_{n} | \eta_{n}|^{2} \frac{1}{2} \frac{ s_{n}^{2} } 
                                               { (1 + s _{n} )^{2} } 
\end{multline}

\begin{multline}
 I  = 
  \sum_{m<n} 
    \frac{ s_{m} s_{n} } { (I/I_{\mathrm{sat}}) ( 1+s_{m} )( 1+s_{n} ) }
    2 \Re\left[ 
        \eta_{m}\eta_{n}^{*}
    \left(
        \frac{ \Delta_{m} \Delta_{n} }{ \Gamma^{2} } 
      + i \frac{ \Delta_{n} }{ 2 \Gamma } 
      - i \frac{ \Delta_{m} }{ 2 \Gamma } 
      + \frac{1}{4}  
    \right) \right] \\
  + \sum_{n}  |\eta_{n}|^{2}
    \frac{1}{2}	\frac{ s_{n} } { ( 1+s_{n} )^{2} }
   + \sum_{n} | \eta_{n}|^{2} \frac{1}{2} \frac{ s_{n}^{2} } 
                                               { (1 + s _{n} )^{2} } 
\end{multline}

With this expression in hand we focus our attention on the terms
$\eta_{m}\eta_{n}^{*}$ and $|\eta_{n}|^{2}$.  We start with the latter 
\begin{equation}
 |\eta_{n}|^{2} =  \frac{\hbar c k \Gamma}{r_{D}^{2}}  
    \frac{9}{24\pi} 
       | (\bv{\varepsilon}_{+}\cdot \bv{\varepsilon}' )
                       (\bv{\varepsilon}^{*}\cdot \bv{\varepsilon}_{-} ) |^{2}
      \langle u | e^{-i(\bv{k}'-\bv{k}) \cdot\bv{r}_{n}} | u'  \rangle
      \langle u' | e^{i(\bv{k}'-\bv{k}) \cdot\bv{r}_{n}} | u  \rangle
\end{equation}
and notice that we have to sum over output polarizations $\bv{\varepsilon}'$
and final center of mass states  $u'$, since our detector does not care about
either. We obtain 
\begin{equation}
\begin{split}
 \sum_{\bv{\varepsilon}' u'}|\eta_{n}|^{2} = & 
    \sum_{ u'} \frac{\hbar c k \Gamma}{r_{D}^{2}}  
    \frac{9}{24\pi} 
      \sum_{\bv{\varepsilon}'} | (\bv{\varepsilon}_{+}\cdot \bv{\varepsilon}' )
                       (\bv{\varepsilon}^{*}\cdot \bv{\varepsilon}_{-} ) |^{2}
      \langle u | e^{-i(\bv{k}'-\bv{k}) \cdot\bv{r}_{n}} | u'  \rangle
      \langle u' | e^{i(\bv{k}'-\bv{k}) \cdot\bv{r}_{n}} | u  \rangle \\
 = & \frac{\hbar c k \Gamma}{r_{D}^{2}}  
    \frac{9}{24\pi} \Lambda 
      \langle u | e^{-i(\bv{k}'-\bv{k}) \cdot\bv{r}_{n}}  e^{i(\bv{k}'-\bv{k}) 
      \cdot\bv{r}_{n}} | u  \rangle \\
 = & \frac{\hbar c k \Gamma}{r_{D}^{2}}  
    \frac{9}{24\pi} \Lambda \\ 
\end{split}
\end{equation}
where we have used the closure relation $\sum{u'}|u'\rangle\langle u'| =
\mathbb{1}$, and have defined for brevity 
\begin{equation}
 \Lambda = 
  \sum_{\bv{\varepsilon}' }
        | (\bv{\varepsilon}_{+}\cdot \bv{\varepsilon}' )
                        (\bv{\varepsilon}\cdot \bv{\varepsilon}_{-} ) |^{2} 
\end{equation} 

Similarly, for $\eta_{m}\eta_{n}^{*}$
\begin{equation}
\begin{split}
 \sum_{\bv{\varepsilon}' u'_{m} u'_{n}} \eta_{m}\eta_{n}^{*} = & 
    \frac{\hbar c k \Gamma}{r_{D}^{2}}  
    \frac{9}{24\pi} \Lambda
 \sum_{u'_{m} u'_{n}} 
      \langle u_{n} | e^{-i(\bv{k}'-\bv{k}) \cdot\bv{r}_{n}} | u'_{n}  \rangle
      \langle u'_{m} | e^{i(\bv{k}'-\bv{k}) \cdot\bv{r}_{m}} | u_{m}  \rangle \\
\end{split}
\end{equation}
In this case we cannot use the closure relation because $n,m$ refer to
different atoms.   We simplify the treatment by considering only final states
for the atom that are the same as the initial state $u'=u$ (these are going to
have the largest matrix elements anyways). In the sum over $u'_{m},u'_{n}$ only
$u'_{m}=u_{m}$ and $u'_{n}=u_{n}$ contribute.  We take the center of mass state
of the atoms to be the ground state of the single lattice site harmonic
oscillator.  This leaves us with 
\begin{equation}
\begin{split}
 \sum_{\bv{\varepsilon}' } \eta_{m}\eta_{n}^{*} = & 
  \frac{\hbar c k \Gamma}{r_{D}^{2}}  
    \frac{9}{24\pi} \Lambda
      \langle 0_{n} | e^{-i(\bv{k}'-\bv{k}) \cdot\bv{r}_{n}} | 0_{n}  \rangle
      \langle 0_{m} | e^{i(\bv{k}'-\bv{k}) \cdot\bv{r}_{m}} | 0_{m}  \rangle \\
\end{split}
\end{equation}

\subsubsection{Debye-Waller factor} 

For each center of mass expectation value we perform a translation $\bv{R}_{n}$
of the coordinate system such that the position of the $n^{\text{th}}$ atom has
a zero expectation value $\langle \bv{r}_{n} \rangle = 0$.  A phase factor
comes out that depends on the position $\bv{R}_{n}$ of the lattice site in
which the atom is located:
\begin{equation}
      \langle 0_{n} | e^{-i(\bv{k}'-\bv{k}) \cdot\bv{r}_{n}} | 0_{n}  \rangle 
    = e^{-i(\bv{k}'-\bv{k}) \cdot\bv{R}_{n}} 
      \langle 0_{n} | e^{-i(\bv{k}'-\bv{k}) \cdot\bv{r}_{n}} | 0_{n}  \rangle
\end{equation} 
We then use the equality $\langle e^{\hat{A}} \rangle = e^{\frac{1}{2} \langle
\hat{A}^{2} \rangle }$, which is valid for a simple harmonic oscillator, where
$\hat{A}$ is any linear combination of displacement and momentum operators of
the oscillator.  This leaves us with
\begin{equation}
\begin{split}
      \langle 0_{n} | e^{-i(\bv{k}'-\bv{k}) \cdot\bv{r}_{n}} | 0_{n}  \rangle 
    = & e^{-i(\bv{k}'-\bv{k}) \cdot\bv{R}_{n}} 
      e^{ -\frac{1}{2} \left\langle 
          [ (\bv{k}'-\bv{k}) \cdot\bv{r}_{n} ]^{2} \right\rangle } \\
    = & e^{ -i \bv{Q} \cdot \bv{R}_{n}} 
      e^{ -\frac{1}{2} \left\langle [ \bv{Q} \cdot\bv{r}_{n} ]^{2} \right\rangle } \\ 
    = & e^{ -i \bv{Q} \cdot \bv{R}_{n}}
      \prod_{i=x,y,z} e^{ - \frac{1}{2}Q_{i}^{2}\langle r_{ni} ^{2} \rangle } \\ 
    = & e^{ -i \bv{Q} \cdot \bv{R}_{n}}
      e^{-W} 
\end{split}
\end{equation} 
where we have defined the momentum transfer $\bv{Q} = \bv{k}' - \bv{k}$,  and
the Debye-Waller factor $e^{-2W}$. 

Putting this back in the expression for $\eta_{m}\eta_{n}^{*}$ we get
\begin{equation}
\begin{split}
 \sum_{\bv{\varepsilon}' } \eta_{m}\eta_{n}^{*} = & 
 \frac{\hbar c k \Gamma}{r_{D}^{2}}  
    \frac{9}{24\pi} \Lambda
       e^{ i \bv{Q}( \bv{R}_{m} - \bv{R}_{n} ) } e^{-2W} 
\end{split}
\end{equation}
And if we now return to the expression for the intensity at the detector we
have 
\begin{multline}
 I  = 
  \sum_{m<n} 
    \frac{ s_{m} s_{n} } 
         { (\iisat) ( 1+s_{m} )( 1+s_{n} ) }
    2 \Re\left[ 
            \frac{\hbar c k \Gamma}{r_{D}^{2}}  
            \frac{9}{24\pi}  \Lambda
               e^{ i \bv{Q}( \bv{R}_{m} - \bv{R}_{n} ) } e^{-2W}  
    \left(
        \frac{ \Delta_{m} \Delta_{n} }{ \Gamma^{2} } 
      + i \frac{ \Delta_{n} }{ 2 \Gamma } 
      - i \frac{ \Delta_{m} }{ 2 \Gamma } 
      + \frac{1}{4}  
    \right) \right] \\ 
  + \sum_{n}  \frac{1}{2}
    \frac{\hbar c k \Gamma}{r_{D}^{2}}  
    \frac{9}{24\pi} \Lambda
    \frac{ s_{n} } { 1 + s_{n} } 
\end{multline}

\begin{multline}
\label{eq:finalIdetector}
 I  =
 \left( 
 \frac{\hbar c k \Gamma}{r_{D}^{2}}  
     \frac{9}{24\pi} \Lambda 
  \right) \times \\
  \sum_{m<n} 
    \frac{ s_{m} s_{n} } 
         { (\iisat) ( 1+s_{m} )( 1+s_{n} ) }
    2 \Re\left[ 
               e^{ i \bv{Q}( \bv{R}_{m} - \bv{R}_{n} ) } e^{-2W}  
    \left(
        \frac{ \Delta_{m} \Delta_{n} }{ \Gamma^{2} } 
      + i \frac{ \Delta_{n} }{ 2 \Gamma } 
      - i \frac{ \Delta_{m} }{ 2 \Gamma } 
      + \frac{1}{4}  
    \right) \right]  
  + \sum_{n}  \frac{1}{2}
    \frac{ s_{n} } { 1 + s_{n} } 
\end{multline}
It is good to see that for a large time-of-flight, where the Debye-Waller
factor goes to zero due to large extent of the expanding atom wavefunctions,
this formula reduces to the the standard uncorrelated scattering for $N$ atoms,
$I=N\rho_{ee}$ with $\rho_{ee} = \frac{1}{2} \frac{s}{1+s}$.  

{\small Note:  To see this more clearly and at the same time check the
prefactors that show up in this expression, we can evaluate the total photon
scattering rate  $\Gamma_{\mathrm{scatt}}= \frac{1}{\hbar c k}\int I r_{D}^{2}
\mathrm{d}\Omega $,  for which we use $\int \Lambda \mathrm{d} \Omega =
\frac{8\pi}{3}$ to obtain
\begin{equation}
\Gamma_{\mathrm{scatt}} =  \Gamma \frac{1}{2} \frac{s}{1+s} = \Gamma \rho_{ee}
\end{equation} 
 
}


\subsection{ Large detuning limit} 

We start from Eq.~(\ref{eq:finalIdetector}) and concentrate on the two sums,
the first of which is 
\begin{equation} 
  \frac{  e^{-2W}}{2\iisat} \Re 
  \sum_{m<n} 
               e^{ i \bv{Q}( \bv{R}_{m} - \bv{R}_{n} ) } 
    \frac{ s_{m} s_{n} } {( 1+s_{m} )( 1+s_{n} ) }
    \left(
         4\Delta_{m} \Delta_{n} 
      + 2i \Delta_{n} 
      - 2i \Delta_{m}
      + 1
    \right)  
\end{equation}
where for simplicity we have now written the detunings in units of $\Gamma$.
We will split this up further into four terms 
\begin{align} 
  \frac{  e^{-2W}}{2\iisat} \Re \sum_{m<n} & 
      e^{ i \bv{Q}( \bv{R}_{m} - \bv{R}_{n} ) } 
      \frac{ s_{m} s_{n} } {( 1+s_{m} )( 1+s_{n} ) } 4 \Delta_{m} \Delta_{n} \\
  \frac{  e^{-2W}}{2\iisat} \Re \sum_{m<n} & 
      e^{ i \bv{Q}( \bv{R}_{m} - \bv{R}_{n} ) } 
      \frac{ s_{m} s_{n} } {( 1+s_{m} )( 1+s_{n} ) } 2 i \Delta_{n}  \\
  -\frac{  e^{-2W}}{2\iisat} \Re \sum_{m<n} & 
      e^{ i \bv{Q}( \bv{R}_{m} - \bv{R}_{n} ) } 
      \frac{ s_{m} s_{n} } {( 1+s_{m} )( 1+s_{n} ) } 2 i \Delta_{m}  \\
  \frac{  e^{-2W}}{2\iisat} \Re \sum_{m<n} & 
      e^{ i \bv{Q}( \bv{R}_{m} - \bv{R}_{n} ) } 
      \frac{ s_{m} s_{n} } {( 1+s_{m} )( 1+s_{n} ) }   
\end{align}

For a detuning such that $4\Delta_{m}^{2}, 4\Delta_{n}^{2} \gg 1 $ we have
\begin{equation}
  \frac{s}{1+s} \approx \frac{2 \iisat }
                             { 4 \Delta^{2} + 2 \iisat }
\end{equation}

and the four terms above go respectively to  
\begin{align} 
     e^{-2W} 2(\iisat)\Re \sum_{m<n} & 
      e^{ i \bv{Q}( \bv{R}_{m} - \bv{R}_{n} ) } 
      \frac{ 4 \Delta_{m} \Delta_{n} } 
           {(4 \Delta_{m}^{2} + 2 \iisat)(4 \Delta_{n}^{2} + 2 \iisat)
 }  \\
     e^{-2W} 2(\iisat)\Re \sum_{m<n} & 
      e^{ i \bv{Q}( \bv{R}_{m} - \bv{R}_{n} ) } 
      \frac{ 2 i \Delta_{n} } 
           {(4 \Delta_{m}^{2} + 2 \iisat)(4 \Delta_{n}^{2} + 2 \iisat)
 }   \\
    -e^{-2W} 2(\iisat) \Re \sum_{m<n} & 
      e^{ i \bv{Q}( \bv{R}_{m} - \bv{R}_{n} ) } 
      \frac{ 2 i \Delta_{m} } 
           {(4 \Delta_{m}^{2} + 2 \iisat)(4 \Delta_{n}^{2} + 2 \iisat)
 }   \\
     e^{-2W} 2(\iisat)\Re \sum_{m<n} & 
      e^{ i \bv{Q}( \bv{R}_{m} - \bv{R}_{n} ) } 
      \frac{ 1} { (4 \Delta_{m}^{2} + 2 \iisat)(4 \Delta_{n}^{2} + 2 \iisat)
}   
\end{align}
%\begin{align} e^{-2W} 2(\iisat) \Re \sum_{m<n} & e^{ i \bv{Q}( \bv{R}_{m} -
%\bv{R}_{n} ) } \frac{1}{ 4 \Delta_{m} \Delta_{n} }  \\ e^{-2W} 2(\iisat) \Re
%\sum_{m<n} & e^{ i \bv{Q}( \bv{R}_{m} - \bv{R}_{n} ) } \frac{i}{ 8 \Delta_{n}
%\Delta_{m}^{2}} \\ - e^{-2W} 2(\iisat) \Re \sum_{m<n} & e^{ i \bv{Q}(
%\bv{R}_{m} - \bv{R}_{n} ) } \frac{i}{ 8 \Delta_{n}^{2} \Delta_{m} } \\ e^{-2W}
%2(\iisat) \Re \sum_{m<n} & e^{ i \bv{Q}( \bv{R}_{m} - \bv{R}_{n} ) } \frac{1}{
%16 \Delta_{m}^{2} \Delta_{n}^{2} } \end{align}
Furthermore, if we detune the light in between the two spin states then we can
use, $\Delta_{m}^{2} = \Delta^{2}$ and  $\Delta_{m} = 2|\Delta|S_{zm}$, where
$S_{zm}=\pm\frac{1}{2}$ is the spin state of the atom in site $m$, to obtain
\begin{align} 
     e^{-2W} 2(\iisat)
      \frac{ 16 \Delta^{2}  } 
           {(4 \Delta^{2} + 2 \iisat)^{2} }  
       \Re \sum_{m<n} & 
      e^{ i \bv{Q}( \bv{R}_{m} - \bv{R}_{n} ) } 
      S_{zm}S_{zn}\\
     e^{-2W} 2(\iisat)
      \frac{ 4 i |\Delta| } 
           {(4 \Delta^{2} + 2 \iisat)^{2} }  
      \Re \sum_{m<n} & 
      e^{ i \bv{Q}( \bv{R}_{m} - \bv{R}_{n} ) } 
      S_{zn}  \\
    -e^{-2W} 2(\iisat) 
      \frac{ 4 i |\Delta|} 
           {(4 \Delta^{2} + 2 \iisat)^{2} }  
      \Re \sum_{m<n} & 
      e^{ i \bv{Q}( \bv{R}_{m} - \bv{R}_{n} ) } 
        S_{zm} \\
     e^{-2W} 2(\iisat)
      \frac{ 1} { (4 \Delta^{2} + 2 \iisat)^{2} }   
       \Re \sum_{m<n} & 
      e^{ i \bv{Q}( \bv{R}_{m} - \bv{R}_{n} ) } 
\end{align}
%\begin{align} e^{-2W} \frac{2\iisat}{ |\Delta|^{2} }  \Re \sum_{m<n} & e^{ i
%\bv{Q}( \bv{R}_{m} - \bv{R}_{n} ) } S_{zm}S_{zn}  \\ e^{-2W} \frac{2\iisat}{ 4
%|\Delta|^{3} }  \Re \sum_{m<n} & e^{ i \bv{Q}( \bv{R}_{m} - \bv{R}_{n} ) } i
%S_{zn} \\ - e^{-2W} \frac{2\iisat}{ 4 |\Delta|^{3} }  \Re \sum_{m<n} & e^{ i
%\bv{Q}( \bv{R}_{m} - \bv{R}_{n} ) } i S_{zm} \\  e^{-2W}
%\frac{2I/I_{\mathrm{sat}}}{ 16 |\Delta|^{4} }  \Re \sum_{m<n} & e^{ i \bv{Q}(
%\bv{R}_{m} - \bv{R}_{n} ) } \end{align}
We identify the first term and the last term as related to the spin structure
factor and crystal structure factor.
%%%%% that appear in Ted's %%%%paper.   We have also two other terms show up
%here, which Ted discards in his %%%%derivation, see more details on
%Sec.~\ref{sec:Ted}.    As was stated at the %%%%beginning of this document,
%our main goal here is to connect the intensity %%%%measured by our cameras
%with the spin-structure factor that is calculated by the %%%%theorists.   
In this last equation we see that the first term, the one related to the spin
structure factor, is going to have the main contribution to the intensity
because it goes as $ |\Delta|^{-2} $, whereas the other terms go as larger
powers of $1/|\Delta|$.  If we neglect terms other than the first one,  we
obtain
\begin{equation}
\begin{split} 
 I  =  & 
 \left( 
 \frac{\hbar c k \Gamma}{r_{D}^{2}}  
     \frac{9}{24\pi} \Lambda 
  \right) \left[
     e^{-2W}
      \frac{ 16 \Delta^{2} (\iisat)  } 
           {(4 \Delta^{2} + 2 \iisat)^{2} }  
       2 \Re \sum_{m<n}  
      e^{ i \bv{Q}( \bv{R}_{m} - \bv{R}_{n} ) } 
      S_{zm}S_{zn}
  + \sum_{n}  \frac{1}{2} \frac{2 \iisat }{ 4 \Delta^{2} + 2 \iisat }
\right]  \\ 
 =  & \left( 
 \frac{\hbar c k \Gamma}{r_{D}^{2}}  
     \frac{9}{24\pi} \Lambda 
  \right) 
  \frac{ \iisat }{ 4 \Delta^{2} + 2 \iisat }
  \left[
      \frac{ e^{-2W}16 \Delta^{2}  } 
           {(4 \Delta^{2} + 2 \iisat) }  
       2 \Re \sum_{m<n}  
      e^{ i \bv{Q}( \bv{R}_{m} - \bv{R}_{n} ) } 
      S_{zm}S_{zn}
  + N 
\right]  \\ 
\end{split}
\end{equation}
We note that, since exchanging indexes in the summand results in its complex
conjugate, $2\Re\sum_{m<n} \equiv \sum_{m\neq n}$.  We also use $\sum_{m\neq n}
\equiv \sum_{mn} - \sum_{m=n}$ to obtain  
\begin{equation}
\begin{split} 
 I 
&  =
  \left( 
 \frac{\hbar c k \Gamma}{r_{D}^{2}}  
     \frac{9}{24\pi} \Lambda 
  \right) 
  \frac{ \iisat }{ 4 \Delta^{2} + 2 \iisat }
  \left[
      \frac{ e^{-2W}16 \Delta^{2}  } 
           {(4 \Delta^{2} + 2 \iisat) } 
   \left( \sum_{mn} - \sum_{m=n} \right) 
      e^{ i \bv{Q}( \bv{R}_{m} - \bv{R}_{n} ) } 
      S_{zm}S_{zn}
  + N 
   \right]  \\ 
&  =
  \left( 
 \frac{\hbar c k \Gamma}{r_{D}^{2}}  
     \frac{9}{24\pi} \Lambda 
  \right) 
  \frac{ \iisat }{ 4 \Delta^{2} + 2 \iisat }
  \left[
      \frac{ e^{-2W}4 \Delta^{2}  } 
           {(4 \Delta^{2} + 2 \iisat) } 
   \left( 4\sum_{mn}  
      e^{ i \bv{Q}( \bv{R}_{m} - \bv{R}_{n} ) } 
      S_{zm}S_{zn}
     - N \right)
  + N 
   \right]  \\ 
\end{split}
\label{eq:iscatt-large-detuning}
\end{equation}
At this point we consider a measurement of the intensity after a large
time-of-flight (TOF), denoted as $\itof$.  After TOF, the Debye-Waller factor
goes to zero due to the expanding size of the atomic wavefunction, so we have 
\begin{equation}
\begin{split} 
 \itof
&  =
  \left( 
 \frac{\hbar c k \Gamma}{r_{D}^{2}}  
     \frac{9}{24\pi} \Lambda 
  \right) 
  \frac{ \iisat }{ 4 \Delta^{2} + 2 \iisat } N 
\end{split}
\end{equation}
\begin{equation}
\begin{split} 
 \frac{I}{\itof} 
&  =
      \frac{ e^{-2W}4 \Delta^{2}  } 
           {(4 \Delta^{2} + 2 \iisat) } 
   \left(
      \frac{4}{N}\sum_{mn}  
      e^{ i \bv{Q}( \bv{R}_{m} - \bv{R}_{n} ) } 
      S_{zm}S_{zn}
     - 1 \right)
  + 1 
\end{split}
\label{eq:IscattQ}
\end{equation}

\subsection{Measurement of the structure factor} 

The theorists in our collaboration calculate the spin structure factor
$S(\bv{Q})$, which is defined as 
\begin{equation}
   S(\bv{Q}) =  
      \frac{4}{N}\sum_{mn}  
      e^{ i \bv{Q}( \bv{R}_{m} - \bv{R}_{n} ) } 
      S_{zm}S_{zn}
\end{equation}
From inspection of Eq.~\ref{eq:IscattQ} we see that a measurement of $I/\itof$
can be related to the spin structure factor by 
\begin{equation}
\begin{split} 
 \frac{I}{\itof} 
&  =
      \frac{ e^{-2W}4 \Delta^{2}  } 
           {(4 \Delta^{2} + 2 \iisat) } 
   \left( S(\bv{Q})
     - 1 \right)
  + 1 
\end{split}
\end{equation}

At this point we introduce the notation
\begin{equation}
 \frac{I}{\itof} \equiv I_{\bv{Q}} 
   \ \ \ \ \ \ \  \ \ \ \ \ \ \ \  
      c \equiv \frac{ e^{-2W}4 \Delta^{2}  } 
           {(4 \Delta^{2} + 2 \iisat) }
   \ \ \ \ \ \ \  \ \ \ \ \ \ \ \ 
    S(\bv{Q}) \equiv S_{\bv{Q}}
\end{equation} 
\begin{equation}
  I_{\bv{Q}} = c ( S_{\bv{Q}} - 1 ) + 1 
\end{equation}
 
If we perform the measurement in a deep enough lattice, then the Debye-Waller
factor $e^{-2W} \rightarrow 1 $ .   Furthermore, if we select an intensity such
that $ \iisat \ll 2\Delta^{2} $  this reduces to an equivalence between the TOF
normalized intensity and the spin structure factor:
\begin{equation}
  I_{\bv{Q}} = S_{\bv{Q}} 
\end{equation}

To compare with the calculations from the theorists we solve for $S_{\bv{Q}}$
and find
\begin{equation} 
  S_{\bv{Q}} = 1 + \left( I_{\bv{Q}}  - 1
\right)/c 
\end{equation}

In our experiment we are using $\iisat \approx 15$ and $\Delta = 6.5$.   We
perform the measurement at a lattice depth of 50$E_{R}$.  For this parameters
we have the following: 
\begin{equation}
\begin{split}
  e^{-2W} & =  0.81\\
  1  +   \frac{\iisat}{2\Delta^{2} } & =  1.18\\  
\end{split}
\end{equation} 

\begin{equation} 
  S_{\bv{Q}} - 1  =   1.45 (I_{t} - 1)
\end{equation}

\subsection{Magneto-association and contributions from doubly occupied sites} 

In our experiment we have the possibility to  magneto-associate (MA) the atoms
in doubly occupied sites prior to a measurement of the scattered light.  The
formed molecules  are insensitive to the probe light and do not scatter any
light at all. In this section we study the implications of MA for the
comparison between experimental measurements and the structure factor as
calculated by the theorists. 

We start from Eq.~\ref{eq:iscatt-large-detuning} and will separate the sums
over atoms in singly and doubly occupied sites.  We have for the intensity
\begin{equation}
\begin{split} 
 I 
&  = \frac{ \itof  }{N} 
  \left[ 4 c 
   \left( \sum_{mn} - \sum_{m=n} \right) 
      e^{ i \bv{Q}( \bv{R}_{m} - \bv{R}_{n} ) } 
      S_{zm}S_{zn}
  + \sum_{n}  
   \right]  \\ 
&  = \frac{ \itof }{N} 
  \left[ 
      4 c \sum_{mn}  
      e^{ i \bv{Q}( \bv{R}_{m} - \bv{R}_{n} ) } 
      S_{zm}S_{zn}
    -  c  \sum_{n} 
  + \sum_{n}  
   \right]  \\ 
\end{split}
\end{equation}

With MA prior to the intensity measurement, there is no contribution to the
scattered light from atoms in doubly occupied sites, so denoting as
$\mathcal{S}$ the set of singly occupied sites, we have 
\begin{equation}
\begin{split} 
 I_{\text{MA}} 
&  = \frac{ \itof }{N} 
  \left[ 
      4 c \sum_{m\in \mathcal{S}}  
      e^{ i \bv{Q} \bv{R}_{m} } S_{zm} 
      \sum_{n\in \mathcal{S} } 
      e^{ -i \bv{Q} \bv{R}_{n} } S_{zn} 
    -  c  \sum_{n\in \mathcal{S}} 
  + \sum_{n \in \mathcal{S}}  
   \right]  \\ 
&  = \frac{ \itof }{N} 
  \left[ 
      4 c \sum_{m\in \mathcal{S}}  
      e^{ i \bv{Q} \bv{R}_{m} } S_{zm} 
      \sum_{n\in \mathcal{S} } 
      e^{ -i \bv{Q} \bv{R}_{n} } S_{zn} 
    -  c N(1-D)  + N(1-D) 
   \right]  \\ 
\end{split}
\end{equation}
where $D$ is the fraction of atoms in doubly occupied sites.  

Denoting the set of atoms in doubly occupied sites as $\mathcal{D}$, we point
out that 
\begin{equation}
 \sum_{m \in \mathcal{D} }  e^{ \pm i \bv{Q} \bv{R}_{m}} S_{zm} = 0 
\end{equation}
since the contributions cancel out in pairs as $   e^{ \pm i \bv{Q} \bv{R}_{m}}
( +1/2 - 1/2 ) = 0 $.  This allows us to remove the $\in \mathcal{S}$
constraint in the remaining sums, 
\begin{equation}
 \sum_{m \in \mathcal{S} }  e^{ \pm i \bv{Q} \bv{R}_{m}} S_{zm} = 
 \sum_{m}  e^{ \pm i \bv{Q} \bv{R}_{m}} S_{zm}  
\end{equation}

and obtain for the intensity after MA 
\begin{equation}
\begin{split} 
 I_{\text{MA}}  
&  = \frac{ \itof }{N} 
  \left[ 
      4 c \sum_{m}  
      e^{ i \bv{Q} \bv{R}_{m} } S_{zm} 
      \sum_{n } 
      e^{ -i \bv{Q} \bv{R}_{n} } S_{zn} 
    -  c N(1-D)  + N(1-D) 
   \right]  \\ 
&  =  \itof 
  \left[ c S_{\bv{Q}}
    -  c (1-D)  + (1-D) 
   \right]  \\ 
\end{split}
\end{equation}

We introduce the simplifying notation
\begin{equation}
 \frac{ I_{\mathrm{MA}} } { \itof } \equiv I_{\bv{Q}m} 
\end{equation}
and simplify for the spin structure factor we obtain 
\begin{equation}
S_{\bv{Q}} = \frac{1}{c}\left( I_{\bv{Q}m}  - 1 \right) 
             - \frac{D}{c}(c-1)  + 1 
\end{equation}

To measure the fraction of atoms in doubly occupied sites, $D$, we can make a
measurement of the scattered intensity after MA and a large TOF.  Combined with
the TOF measurement without MA we have 
\begin{equation}
  D = 1 - \frac{I_{\text{MA},\infty}}{ \itof } \equiv 1 - I_{s} 
\end{equation} 
giving finally 
\begin{equation}
S_{\bv{Q}} = \frac{1}{c}\left( I_{\bv{Q}m}  - 1 \right) 
            + \frac{c-1}{c}\left(  I_{s}   - 1\right)   + 1 
\end{equation}
where we recall 
\begin{equation}
 I_{\bv{Q}m} = \frac{ I_{\text{MA}}}{\itof}  
 \ \ \ \ \ \ \ \ \ \ \ 
 I_{s} = \frac{ I_{\text{MA},\infty} } {\itof} 
\end{equation}

\subsection{Data at short time-of-flight} 

In the above section we considered $\itof$, a measurement of the intensity with
a time-of-flight long enough such that the Debye-Waller factor $e^{-2W}
\rightarrow 0 $.   For some of the data we have taken, the TOF is 6~$\mu$s,
which in some cases is not long enough for the Debye-Waller factor be
negligible.  In this section we look at the determination of $S_{\bv{Q}}$ with
time-of-flight data for which one cannot assume $e^{-2W} \rightarrow 0 $.  

We recall that the Debye-Waller factor is defined as
\begin{equation}
    e^{-2W} = 
      \prod_{i=x,y,z} e^{ -Q_{i}^{2}\langle r_{i} ^{2} \rangle } 
\end{equation}
where $Q_{i}$ is the $i^{\text{th}}$ component of the momentum transfer $\bv{Q}
= \bv{k}' - \bv{k}$,  with $\bv{k}'$ and $\bv{k}$ the wave vectors of the
outgoing and incoming light respectively.  The time-of-flight dependence of the
Debye-Waller factor is through the expanding size of the atomic wavefunctions.
For an in-situ picture $\langle r_{i}^{2}\rangle$ corresponds to the position
spread of the lattice site's harmonic oscillator ground state.  After the atoms
are released from the lattice the spread of the wavefunction satisfies 
\begin{equation}
\begin{split} 
  \langle r_{i}^{2} \rangle_{t} = &
   \langle r_{i}^{2} \rangle_{0} + 
  \frac{t^{2}}{m^{2}} \langle p_{i}^{2} \rangle_{0} \\
  = &  \langle r_{i}^{2} \rangle_{0} + 
  \frac{t^{2}}{m^{2}} \frac{\hbar^{2}}{4  \langle r_{i}^{2} \rangle_{0} } \\
\end{split}
\end{equation} 

In a harmonic oscillator potential we have 
\begin{equation}
    \langle r^{2} \rangle = \frac{\hbar}{2 m \omega}
\end{equation}
and since in a lattice of depth $V_{0}$ recoils,  $\hbar \omega = 2 E_{R}
\sqrt{V_{0}}$  
\begin{equation}
    \langle r^{2} \rangle = \frac{a^{2}}{ 2 \pi^{2} \sqrt{V_{0}} }
\end{equation}
where $a$ is the lattice spacing. 

We then have for the Debye-Waller factor as a function of time-of-flight
\begin{equation}
\begin{split}
    [e^{-2W}]_{t} = &  
      \prod_{i=x,y,z} \exp\left[  -Q_{i}^{2} 
      \left(  \frac{a^{2}}{ 2 \pi^{2} \sqrt{V_{0}}} 
            + \frac{ h^{2} \sqrt{V_{0}} }{8 a^{2}} 
              \frac{ t^{2}}{m^{2} }\right) \right] \\
    =& [e^{-2W}]_{0} \prod_{i=x,y,z} \exp\left[  -Q_{i}^{2} 
            \frac{ h^{2} \sqrt{V_{0}} }{8 a^{2}} 
              \frac{ t^{2}}{m^{2} } \right] \\
    =& [e^{-2W}]_{0} \prod_{i=x,y,z} \exp\left[ -\frac{\sqrt{V_{0}}}{2}
           \left( \frac{ Q_{i} h }{2 m a  } \right)^{2}  t^{2}   \right] \\
    =& [e^{-2W}]_{0} \exp\left[ -\frac{\sqrt{V_{0}}}{2}
           \left( \frac{ |\bv{Q}| h }{2 m a  } \right)^{2}  t^{2}   \right] \\
\end{split}
\end{equation}

We now define the time dependent correction factor 
\begin{equation}
   c(t) =   \frac{ [e^{-2W}]_{t} 4 \Delta^{2}  } 
           {4 \Delta^{2} + 2 \iisat }  
\end{equation}
The measured intensity after time-of-flight $t$ is 
\begin{equation}
\begin{split} 
 \frac{I(t)}{\itof} 
&  = c(t) ( S_{\bv{Q}} - 1 ) + 1 
\end{split}
\end{equation}
Consider two measurements, one at $t=0$, the other one at $t=T$,
\begin{equation}
 \frac{ I(t=0) }{\itof} =  c(0)( S_{\bv{Q}} -1) + 1 
 \ \ \ \ \ \ \ \ \ \ 
  \frac{ I(t=T) }{\itof} =  c(T)( S_{\bv{Q}} -1) + 1 
\end{equation}
\begin{equation} 
 \frac{ I(t=0)}{ I(t=T)} = 
 \frac{ c(0)( S_{\bv{Q}} - 1 ) + 1}{ c(T)( S_{\bv{Q}} - 1) + 1} 
      \equiv I_{\bv{Q}}(T) 
\end{equation}
which can be solved to give
\begin{equation} 
  S_{\bv{Q}} = \frac{ 1 - c(0) - I_{\bv{Q}}(T)( 1- c(T))}
                    { I_{\bv{Q}}(T)c(T) - c(0) } 
\end{equation}

\subsection{Simultaneous measurement of the structure factor for two different
values of $\bv{Q}$}

In the sections above we have shown how to measure the spin structure factor
$S_{\bv{Q}}$  by measuring the scattered intensity in-situ and after
time-of-flight.  An issue that arises with this measurement is that the in-situ
and TOF measurements require two different realizations of the experiment.  

In our setup we have the possibility of measuring the scattered intensity at
two different output angles.  One of the angles corresponds to a momentum
transfer $\bv{Q} = \frac{2\pi}{a} ( \frac{1}{2}\, \frac{1}{2}\, \frac{1}{2} )
\equiv \bv{\pi}$,  which offers the possibility of measuring the staggered
magnetization of the system, i.e. the degree of antiferromagnetic ordering.
The other angle corresponds to a momentum transfer $\bv{Q} = \frac{2\pi}{a}(
0.4,\, -0.1,\, -0.04 ) \equiv \bv{\theta}$.  The structure factor can also be
measured at this value of $\bv{Q}$ and compared with theoretical calculations.
Also, this momentum transfer does not measure the staggered magnetization, so
it may be used as a normalization for the measurement at $\bv{Q}=\bv{\pi}$

For the measurements at $\bv{\pi}$ and $\bv{\theta}$ we have (without
magneto-association)
\begin{equation}
   S_{\bv{\pi}} = (I_{\bv{\pi}} - 1 + c)/c  
  \ \ \ \ \ \ \ \ \ \ 
   S_{\bv{\theta}} = (I_{\bv{\theta}} - 1 + c)/c  
\end{equation}
\begin{equation}
  \frac{ S_{\bv{\pi}} }{ S_{\bv{\theta}}} = 
  \frac{ I_{\bv{\pi}} - 1 + c }{ I_{\bv{\theta}} - 1 + c } 
\end{equation}
We recall that $I_{\bv{Q}} = I(\bv{Q}) / \itof (\bv{Q})$.  To make use of the
simultaneous measurement of $I$ at $\bv{Q}= \bv{\pi}, \bv{\theta}$ we define 
\begin{equation}
  \left. 
  \left( \frac{ I(\bv{\pi})}{I(\bv{\theta})} \right)  \middle/  
  \left( \frac{ \itof(\bv{\pi})}{\itof(\bv{\theta})} \right)  
  \right.  \equiv I_{\bv{\pi}/\bv{\theta}}
\end{equation}
This measurement consists of an in-situ measurement of the ratio
$I(\bv{\pi})/I(\bv{\theta})$  and a TOF measurement of the same ratio
in a subsequent realization of the experiment.  We then have 
\begin{equation}
   S_{\bv{\pi}}  = S_{\bv{\theta}}
  \frac{ I_{\bv{\pi}/\bv{\theta}} - ( 1 - c )/I_{\bv{\theta}} }
        { 1 - (1 - c)/I_{\bv{\theta}} } 
\end{equation}
This measurement is an alternative to the measurement of $S_{\bv{\pi}}$
directly from $I_{\bv{\pi}}$,  it offers the added advantage that the recording
of the ratio $I(\bv{\pi})/I(\bv{\theta})$ in a single shot may result in a
reduction of the noise in the signal.   

From what we have seen so far, the noise in the signal is dominated by the
repeatability of the experimental realizations.  That is, the noise exceeds the
detection noise ( variance of the intensity measurement without atoms), and it
also exceeds the shot to shot variation from atom number and probe
intensity/polarization fluctuations.    With this realization-dominated noise, a
significant reduction in the error associated with $S_{\bv{\pi}}$ is not
expected with the simultaneous measurement.  Nevertheless, this alternate
measure should be consistent with $S_{\bv{\pi}}$ as obtained directly from
$I_{\bv{\pi}}$.  

 

\bibliographystyle{osa}
\bibliography{bragg}

\end{document}




