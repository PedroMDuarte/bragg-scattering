\documentclass[11pt,letter]{article}
\usepackage[top=0.65in,bottom=0.9in,left=0.85in,right=0.85in]{geometry}

%\def\baselinestretch{1.25}
\def\baselinestretch{1.0}

\usepackage[greek, english]{babel}
\usepackage{multicol}

% The use of the times package forces the use of the type-1 times
% roman font, but the times roman font does not look nice.
% Besides the times roman font still does not print correctly on
% the dopy printer.
%\usepackage{times}


\usepackage{fancyhdr}
\usepackage[dvips]{color}
\usepackage{amsmath}
\usepackage{bbold}

\newcommand{\bv}[1]{\ensuremath{\mathbf{#1}}}
\newcommand{\Lc}{\ensuremath{L_{\mathrm{c}}}}
\newcommand{\dsig}[1]{\ensuremath{ \frac{ d\,\sigma_{#1} }{d\,\Omega} }}

\begin{document}

\section{ Calculation of scattering cross section}

The differential scattering cross section has two componets, one which is due
to elastic scattering of photons (the delta-function peak in the fluorescence
spectrum) and the other one which is due to inelastic scattering.   

The elastic part is derived in Ted's paper:
\begin{equation}
\begin{split}
\dsig{E} =&   \frac{9}{4k^{2}} 
              \sum_{\lambda_{f}}  | (\bv{e}_{\bv{k}_{f} \lambda_{f}}^{*} \cdot \bv{e}_{m} ) 
                                   (\bv{e}_{m}^{*} \cdot \bv{e}_{\bv{k}_{i} \lambda_{i}}^{*}  ) 
                                 |^{2} \\
          &   \times \sum_{ \sigma,\sigma', j, j' } [ \langle \hat{n}_{j\sigma}\hat{n}_{j'\sigma'}
              e^{ i \bv{K} \cdot ( \hat{\bv{r}}_{j} - \hat{\bv{r}}_{j'} )  } \rangle 
              \bar{f}_{\sigma} {\bar{f}_{\sigma'}}^{*} ] 
\end{split}
\end{equation}

The following notation is used: 

\begin{center}
  \begin{tabular}{ r  l  }
    $k$ & Magnitude of $k$-vector for incoming photon,  $\frac{2\pi}{671\,\mathrm{nm}}$   \\ 
    $\bv{e}_{\bv{k}_{f} \lambda_{f}}$  & Polarization vector of outgoing photon  \\ 
    $\bv{e}_{\bv{k}_{i} \lambda_{i}}$  & Polarization vector of incoming photon  \\ 
    $\bv{e}_{m}$  & Polarization that couples to optical transition  \\ 
    $\hat{n}_{j\sigma}$ & Number operator at site $j$ for atoms in state $\sigma$   \\ 
    $\bv{K}$ &  Momentum transfer of light scattering, $\bv{k}_{f} - \bv{k}_{i}$  \\ 
    $\bar{f}_{\sigma}$ & Detuning dependent portion of the scattering amplitude, $\frac{\Gamma/2}{\Delta_{\sigma} + i \Gamma/2 }$ \\ 
  \end{tabular}
\end{center}

In addition to the elastic part, the inelastic scattering cross section for the
sample has to be considered.   This can be obtained from the cross section of a
single atom and and then multiplied by the number of atoms in the sample.
Interference effects are not present for the inelastic component.   

\begin{equation}
\begin{split}
\dsig{I} =&  \frac{9}{4k^{2}} 
              \sum_{\lambda_{f}}  | (\bv{e}_{\bv{k}_{f} \lambda_{f}}^{*} \cdot \bv{e}_{m} ) 
                                   (\bv{e}_{m}^{*} \cdot \bv{e}_{\bv{k}_{i} \lambda_{i}}^{*}  ) 
                                 |^{2} 
              \sum_{j\sigma}
                     \frac{\langle \hat{n}_{\sigma} \rangle}{1+ 4\Delta_{\sigma}^{2} + 2 I / I_{\mathrm{sat}} } 
\end{split}
\end{equation}

Finally we abbreviate the sum over final polarizations as 
\begin{equation} 
 \Lambda =       \sum_{\lambda_{f}}  | (\bv{e}_{\bv{k}_{f} \lambda_{f}}^{*} \cdot \bv{e}_{m} ) 
                                   (\bv{e}_{m}^{*} \cdot \bv{e}_{\bv{k}_{i} \lambda_{i}}^{*}  ) 
                                 |^{2}
\end{equation}
to obtain 

\begin{equation}
\begin{split}
\dsig{E} =&   \frac{9\Lambda}{4k^{2}} 
               \sum_{ \sigma,\sigma', j, j' } [ \langle \hat{n}_{j\sigma}\hat{n}_{j'\sigma'}
              e^{ i \bv{K} \cdot ( \hat{\bv{r}}_{j} - \hat{\bv{r}}_{j'} )  } \rangle 
              \bar{f}_{\sigma} {\bar{f}_{\sigma'}}^{*} ] \\  
\dsig{I} =&  \frac{9\Lambda}{4k^{2}} 
              \sum_{j\sigma}
                     \frac{\langle \hat{n}_{\sigma} \rangle}{1+ 4\Delta_{\sigma}^{2} + 2 I / I_{\mathrm{sat}} } 
\end{split}
\end{equation}

\subsection{ Crystal and magnetic structure factor } 

We will begin by disecting the sum that appears in the elastic cross section.   The thermal average factorizes and 
\begin{equation}
\begin{split}
\langle  \hat{n}_{j\sigma}\hat{n}_{j'\sigma'} \rangle = &\, \langle (\frac{1}{2}  + \sigma\hat{S}_{zj} )( \frac{1}{2} + \sigma'\hat{S}_{j'} ) \rangle \\
       = &\,  \frac{1}{4} + \frac{1}{2} \langle \sigma \hat{S}_{zj} \rangle +
\frac{1}{2} \langle \sigma'\hat{S}_{zj'} \rangle + \langle \sigma\sigma'\hat{S}_{zj}\hat{S}_{zj'}
\rangle \\
\end{split} 
\end{equation}
For this last step
\begin{center}
  \begin{tabular}{ c  }
    $\hat{n}_{i\uparrow} + \hat{n}_{i\downarrow} = 1$   \\ 
    $\sigma = \pm 1$ \\
    $\hat{S}_{zi} = \frac{1}{2}( \hat{n}_{i\uparrow} - \hat{n}_{i\downarrow} )$ 
  \end{tabular}
\end{center}
We can manually perform the sums over $\sigma\sigma'$ for each of this four terms and define $\alpha$, $\beta$, and $\kappa$,
\begin{equation}
\begin{split}
\frac{1}{4} \sum_{\sigma\sigma'} \bar{f}_{\sigma} \bar{f}_{\sigma'}^{*} = &\, 
            \frac{1}{4}(\bar{f}_{\uparrow} + \bar{f}_{\downarrow})( \bar{f}_{\uparrow}^{*} + \bar{f}_{\downarrow}^{*} ) \\
       =&\, \frac{1}{4} | \bar{f}_{\uparrow} + \bar{f}_{\downarrow} | ^{2} \equiv \alpha
\end{split} 
\end{equation} 

\begin{equation}
\begin{split}
\frac{1}{2} \sum_{\sigma\sigma'} \sigma'\bar{f}_{\sigma} \bar{f}_{\sigma'}^{*} = &\,  
  \frac{1}{2} ( 
            - \bar{f}_{\downarrow}\bar{f}_{\downarrow}^{*} 
            + \bar{f}_{\downarrow}\bar{f}_{\uparrow}^{*} 
            - \bar{f}_{\uparrow}\bar{f}_{\downarrow}^{*} 
            + \bar{f}_{\uparrow}\bar{f}_{\uparrow}^{*} ) \equiv \kappa 
\end{split} 
\end{equation} 

\begin{equation}
\begin{split}
\frac{1}{2} \sum_{\sigma\sigma'} \sigma \bar{f}_{\sigma} \bar{f}_{\sigma'}^{*} = &\, 
  \frac{1}{2} ( 
            - \bar{f}_{\downarrow}\bar{f}_{\downarrow}^{*} 
            - \bar{f}_{\downarrow}\bar{f}_{\uparrow}^{*} 
            + \bar{f}_{\uparrow}\bar{f}_{\downarrow}^{*} 
            + \bar{f}_{\uparrow}\bar{f}_{\uparrow}^{*} ) \equiv \kappa^{*}
\end{split} 
\end{equation} 

\begin{equation}
\begin{split}
\sum_{\sigma\sigma'} \sigma\sigma' \bar{f}_{\sigma} \bar{f}_{\sigma'}^{*} = &\, 
            (\bar{f}_{\uparrow} + \bar{f}_{\downarrow})( \bar{f}_{\uparrow}^{*} + \bar{f}_{\downarrow}^{*} ) \\
       =&\,  | \bar{f}_{\uparrow} - \bar{f}_{\downarrow} | ^{2} \equiv \beta
\end{split} 
\end{equation}

The elastic cross section is then 
\begin{equation}
\begin{split}
\dsig{E} =&   \frac{9\Lambda}{4k^{2}} 
               \sum_{ j j' }  \langle 
              e^{ i \bv{K} \cdot ( \hat{\bv{r}}_{j} - \hat{\bv{r}}_{j'} )  } \rangle 
             \left( \alpha + \langle \hat{S}_{zj}\rangle \kappa + \langle \hat{S}_{zj'} \rangle \kappa^{*} 
                     + \langle \hat{S}_{zj} \hat{S}_{zj'} \rangle \beta \right)   
\end{split}
\end{equation}

\subsection{Debye-Waller factor} 

The position operators $\hat{\bv{r}}_{j}$,$\hat{\bv{r}}_{j'}$  inside the exponential in the thermal average can be replaced by the position vector of the lattice sites plus a displacement operator with respect to the lattice site.   
\begin{equation}
\begin{split} 
\langle e^{ i \bv{K} \cdot (  \hat{\bv{r}}_{j} - \hat{\bv{r}}_{j'} )  } \rangle = &   
e^{i \bv{K} \cdot ( \bv{R}_{j} - \bv{R}_{j'} ) } 
\langle e^{ i \bv{K} \cdot (  \Delta\hat{\bv{r}}_{j} - \Delta\hat{\bv{r}}_{j'} )  } \rangle \\   
= & e^{i \bv{K} \cdot ( \bv{R}_{j} - \bv{R}_{j'} ) } 
\langle e^{ i \bv{K} \cdot (  \Delta\hat{\bv{r}}_{j} - \Delta\hat{\bv{r}}_{j'} )  } \rangle \\ 
= & e^{i \bv{K} \cdot ( \bv{R}_{j} - \bv{R}_{j'} ) } 
\langle e^{ i \bv{K} \cdot  \Delta\hat{\bv{r}}_{j}} \rangle \langle    e^{ -i \bv{K} \cdot  \Delta\hat{\bv{r}}_{j'}   } \rangle \\ 
\end{split}
\end{equation} 
The last step follows since the operators $\Delta\hat{\bv{r}}_{j}$ and
$\Delta\hat{\bv{r}_{j'}}$  act on different particles.
The equality $\langle e^{\hat{A}} \rangle = e^{\frac{1}{2} \langle \hat{A}^{2} \rangle }$ is used here, 
which is valid for a simple harmonic osscilator where $\hat{A}$ is any linear
combination of displacement and momentum operators of the oscillator.  This leaves us with 
\begin{equation}
\begin{split} 
\langle e^{ i \bv{K} \cdot (  \hat{\bv{r}}_{j} - \hat{\bv{r}}_{j'} )  } \rangle = &   
 e^{i \bv{K} \cdot ( \bv{R}_{j} - \bv{R}_{j'} ) } 
 e^{ -\frac{1}{2} \langle (\bv{K} \cdot  \Delta\hat{\bv{r}}_{j})^{2} \rangle }
 e^{ -\frac{1}{2} \langle (\bv{K} \cdot  \Delta\hat{\bv{r}}_{j'})^{2}  \rangle }  \\
 = &  
 e^{i \bv{K} \cdot ( \bv{R}_{j} - \bv{R}_{j'} ) } 
 e^{ -\langle (\bv{K} \cdot  \Delta\hat{\bv{r}})^{2}  \rangle } \\
 = &  
 e^{i \bv{K} \cdot ( \bv{R}_{j} - \bv{R}_{j'} ) } 
 e^{ -2W} 
\end{split}
\end{equation} 
In the last step we made use of the fact that in our sample all the atoms are
in the same harmonic oscillator state, so the expectation value is be
independent of $j$.   The second exponential with the expectation value is the
Debye-Waller factor, generally written as $e^{-2W}$.  Expanding the dot product inside the Debye-Waller exponential leaves us with
\begin{equation}
\begin{split} 
\langle e^{ i \bv{K} \cdot (  \hat{\bv{r}}_{j} - \hat{\bv{r}}_{j'} )  } \rangle  
 = &  
 e^{i \bv{K} \cdot ( \bv{R}_{j} - \bv{R}_{j'} ) }
 \prod_{i=x,y,z}  e^{ - K_{i}^{2} \langle \Delta r_{i}^{2} \rangle  } 
\end{split}
\end{equation}
In an isotropic lattice all three expectation values are the same and equal to  
\begin{equation}
\langle \Delta r_{i}^{2} \rangle = \frac{\hbar}{m\omega}\left(n+\frac{1}{2}  \right) = \frac{\lambda^{2}}{8 \pi^{2}\sqrt{V_{0}}} ( 1 + 2n) 
\end{equation}
so the Debye-Waller factor is 
\begin{equation}
e^{-2W} = \exp\left[-K^{2} \frac{\lambda^{2}}{8\pi^{2} \sqrt{V_{0}}} ( 1+2n) \right]
\end{equation}
In our case, where we only occupy the first band of the lattice then $n=0$, and  
\begin{equation}
e^{-2W} = \exp\left[-K^{2} \frac{\lambda^{2}}{8\pi^{2} \sqrt{V_{0}}} \right]
\end{equation}

\subsection{Crystal and Structure factors }

Putting it all back together 
\begin{multline}
\dsig{E} =  \frac{9\Lambda}{4k^{2}} e^{-2W} \left(
               \alpha \sum_{ j j' }      e^{i\bv{K}\cdot(\bv{R}_{j}-\bv{R}_{j'})}
            +  \kappa \sum_{ j j' } \langle \hat{S}_{zj} \rangle e^{i\bv{K}\cdot(\bv{R}_{j}-\bv{R}_{j'})} \right. \\
  \left.    +  \kappa^{*} \sum_{ j j' } \langle \hat{S}_{zj'} \rangle e^{i\bv{K}\cdot(\bv{R}_{j}-\bv{R}_{j'})}
            +  \beta \sum_{j j'} \langle \hat{S}_{zj} \hat{S}_{zj'} \rangle e^{i\bv{K}\cdot(\bv{R}_{j}-\bv{R}_{j'})}  \right) 
\end{multline}

The two terms in the center can be simplified by noting that as $N$ gets larger one of the sums approaches a delta-function,  
\begin{equation}
\sum_{j} e^{i \bv{K} \cdot \bv{R}_{j} }  =  \sum_{\bv{n} \in \mathbb{Z}^{3} } \delta \left( \bv{n} - \bv{K}\frac{a}{2\pi} \right) 
\end{equation}
so those two terms will be zero unless $\bv{K}$ is zero or a reciprocal lattice vector.   Even if $\bv{K}=0$ we would then be left with a sum over $\langle\hat{S}_{zj}\rangle$, which is zero in our case since we have a sample without spin imbalance.   Since we cannot shine light in or image along a lattice vector then the two terms in the center can be ignored, leaving us with 
\begin{equation}
\begin{split}
\dsig{E} =&  
             \frac{9\Lambda}{4k^{2}} e^{-2W} \left(
               \alpha \sum_{ j j' }      e^{i\bv{K}\cdot(\bv{R}_{j}-\bv{R}_{j'})}
	    +  \beta \sum_{j j'} \langle \hat{S}_{zj} \hat{S}_{zj'} \rangle
            e^{i\bv{K}\cdot(\bv{R}_{j}-\bv{R}_{j'})}  \right) \\
         =&
             \frac{9\Lambda}{4k^{2}} e^{-2W} 
             \left(
               \alpha C(\bv{K})  
	    +  \beta S(\bv{K})  \right)
\end{split} 
\end{equation}
Where in the last step we have the defined the crystal and magnetic structure factors. 

\subsection{Coherent and Incoherent scattering}

\begin{equation}
\begin{split} 
\dsig{} = & \dsig{E} + \dsig{I}  \\
        = &\frac{9\Lambda}{4k^{2}} 
          \left(
               \alpha C(\bv{K})  
	    +  \beta S(\bv{K}) 
            +  \sum_{j\sigma}
                     \frac{\langle \hat{n}_{\sigma} \rangle}{1+ 4\Delta_{\sigma}^{2} + 2 I / I_{\mathrm{sat}} } 
          \right)
\end{split}
\end{equation}
 
\subsection{ Crystal structure factor dependence on \bv{Q}}

\begin{equation}
C(\bv{Q}) = \sum_{jk} e^{i \bv{Q} \cdot (\bv{R}_{j} - \bv{R}_{k} )} 
\end{equation}

We can make the substitution $\bv{r}_{k} = \bv{R}_{j} -
\bv{R}_{k}$.  For an infinite crystal, the sum over all pairs $jk$ may be replaced by the sum over all
sites $j$ and then sum over all $\bv{r}_{k}$.
 
\begin{equation}
C(\bv{Q}) = N\sum_{k} e^{i \bv{Q} \cdot \bv{r}_{k}} 
\end{equation}

\subsection{Magnetic structure factor dependence on \bv{Q}}

\begin{equation}
S(\bv{Q}) = \sum_{jk} e^{i \bv{Q} \cdot (\bv{R}_{j} - \bv{R}_{k} )} \langle S_{zj} S_{zk} \rangle
\end{equation}

In the zero temperature AFM state there is a staggered magnetization, such that 

\begin{equation}
\langle S_{zj}S_{zk} \rangle = e^{i \bv{q} \cdot (\bv{R}_{j} - \bv{R}_{k} )} 
\ \ \ \mathrm{where} \ \ \  
\bv{q} = \frac{2\pi}{a} \left( \frac{1}{2}\, \frac{1}{2}\, \frac{1}{2} \right)
\end{equation}

At finite temperature, the staggered magnetization will have a finite correlation length \Lc, which results in 

\begin{equation}
\langle S_{zj}S_{zk} \rangle = e^{i \bv{q} \cdot (\bv{R}_{j} - \bv{R}_{k} )} e^{-|\bv{R}_{j} -\bv{R}_{k} |/ \Lc }
 \ \ \ \mathrm{where} \ \ \   
\bv{q} = \frac{2\pi}{a} \left( \frac{1}{2}\, \frac{1}{2}\, \frac{1}{2} \right)
\end{equation}

\begin{equation}
S(\bv{Q}) = \sum_{jk} e^{i (\bv{Q} + \bv{q} )\cdot (\bv{R}_{j} - \bv{R}_{k} )} e^{-|\bv{R}_{j} -\bv{R}_{k} |/ \Lc }
\end{equation}

At this point we can make the substitution $\bv{r}_{k} = \bv{R}_{j} -
\bv{R}_{k}$.  For an infinite crystal, the sum over all pairs $jk$ may be replaced by the sum over all
sites $j$ and then sum over all $\bv{r}_{k}$. 

\begin{equation}
S(\bv{Q}) = \sum_{jk} e^{i (\bv{Q} + \bv{q} )\cdot \bv{r}_{k} } e^{-r_{k}/ \Lc } = N \sum_{k} e^{i (\bv{Q} + \bv{q} )\cdot \bv{r}_{k} } e^{-r_{k}/ \Lc }
\end{equation}

\end{document}




