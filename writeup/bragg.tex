\documentclass[11pt,letter]{article}
\usepackage[top=0.65in,bottom=0.9in,left=0.85in,right=0.85in]{geometry}

%\def\baselinestretch{1.25}
\def\baselinestretch{1.0}

\usepackage[greek, english]{babel}
\usepackage{multicol}

% The use of the times package forces the use of the type-1 times
% roman font, but the times roman font does not look nice.
% Besides the times roman font still does not print correctly on
% the dopy printer.
%\usepackage{times}


\usepackage{fancyhdr}
\usepackage[dvips]{color}
\usepackage{amsmath}
\usepackage{bm}
\usepackage{bbold}

\newcommand{\bv}[1]{\ensuremath{\bm{#1}}}
\newcommand{\Lc}{\ensuremath{L_{\mathrm{c}}}}
\newcommand{\dsig}[1]{\ensuremath{ \frac{ d\,\sigma_{#1} }{d\,\Omega} }}

\begin{document}

\section{Scattering of light by an array of atoms}

In our experiment we observe the scattering of photons from atoms confined in an optical lattice, here we treat this situation by obtaining the the field scattered from a single atom and then summing the field contributions from all the atoms coherently at the location of our detector.  The main goal of this document is to find the connection between the intensity that we measure in our cameras and the spin structure factor as it is calculated by the theorists in our collaboration.  

\subsection{Coherent and Incoherent scattering}

To calculate the scattered field, one uses the source-field expression, which relates the radiated field to the emitting dipole moment, this 
is derived in the standard textbooks~\cite{loudon2000quantum,cohen1998atom}.
The field at the position of the detector $\bv{r}_{D}$ is given by
\begin{equation}
E^{(+)}( \bv{r}_{D}, t) = \eta 
    e^{- i \omega_{L} ( t - r_{D}/c) } S_{-}\left(t - \frac{ r_{D} }{c} \right) 
\end{equation} 
where $\eta$ is a proportionality factor that we will address later on.  

The time-averaged intensity at the detector is 
\begin{equation}
\label{eq:Idef}
\begin{split}
\langle I (t) \rangle = & \langle E^{(-)}(\bv{r}_{D}, t) E^{(+)}(\bv{r}_{D}, t) \rangle \\
          = & |\eta|^{2} \langle S_{+}(t-r_{D}/c)S_{-}(t-r_{D}/c) \rangle  \\
          = & |\eta|^{2} \rho_{ee}( t- r_{D}/c )  \\
          = & |\eta|^{2} \rho_{ee}
\end{split} 
\end{equation}
Where in the last step the time dependence is dropped since we are interested in the steady state solution of $\rho_{ee}$.

We can elucidate the coherence properties of the scattered light if we rewrite the $S_{\pm}$ operators as 
\begin{equation} 
    S_{\pm}( t - r_{D}/c) = \langle S_{\pm}(t-r_{D}/c) \rangle + \delta S_{\pm}(t-r_{D}/c) 
\end{equation} 
at the same time defining the difference $\delta S$ between $S_{\pm}$ and its
average value.   Writing $S$ this way will allow us to distinguish between two
components in the radiated light,  the radiation of the average dipole $\langle
S_{\pm}\rangle$ which is the radiation of a classical oscillating dipole with
a phase that is well defined relative to the incident laser field, and the
radiation form the $\delta S_{\pm}$ component which does not have a phase that
is well defined relative to the incident field because this radiation comes
form the fluctuating part of the atomic dipole.  Dropping the time-dependencies
(since we are interested in the steady-state solution only) we have
\begin{equation} 
I  = \eta^{2} \langle S_{+}\rangle \langle S_{-} \rangle + \eta^{2} \langle \delta S_{+} \delta S_{-} \rangle 
\end{equation}
where we have used the fact that by definition $\langle \delta S_{\pm} \rangle = 0$.The first and second terms of this equation are the coherent and incoherent intensity which can be calculated by using the steady-state solutions to the optical Bloch equations
\begin{equation}
\begin{split} 
    \frac{1}{\eta^{2}} \langle I_{\mathrm{coh}} \rangle = &
        \frac{1}{2} \frac{s}{(1+s)^{2} } 
      = \rho_{ee}  \frac{1}{1+s} \\
    \frac{1}{\eta^{2}} \langle I_{\mathrm{incoh}} \rangle = & 
        \langle S_{+}S_{-} \rangle - |\langle S_{+} \rangle |^{2} \\
      =&\frac{1}{2} \frac{s^{2}}{(1+s)^{2}} = \rho_{ee} \frac{s}{1+s}
\end{split}
\end{equation}
Here $s$ is the saturation parameter
\begin{equation}
s = \frac{ 2  I_{\mathrm{p}}/I_{\mathrm{sat}} } { 1 + 4 \Delta^{2} } 
\end{equation}
with $I_{\mathrm{p}}$ being the intensity of the probe beam.  


\subsection{Resonant scattering cross-section}

Now we will turn onto the evaluation of the proportionality factor $\eta$ which
will contain the angular dependence of the scattered intensity. This is done by considering the  
transition
matrix element between the following inital and final states \begin{equation}
\begin{split}
    | \varphi_{i} \rangle = & | g ; \bv{k}\bv{\varepsilon} \rangle \\
    | \varphi_{f} \rangle = & | g ; \bv{k}'\bv{\varepsilon}' \rangle \\
\end{split}
\end{equation} 
The transition rate to from $i\rightarrow f$ is given by 
\begin{equation}
    \label{eq:transitionRate}
    w_{fi} = \frac{2\pi}{\hbar} | \mathcal{T}_{fi} |^{2} \delta(E_{f}-E_{i})
\end{equation}
Where we use the notation in \cite{cohen1998atom}, and $\mathcal{T}_{fi}$ is given by
\begin{equation}
    \mathcal{T}_{fi} = \frac{  
        \langle g; \bv{k}'\bv{\varepsilon}'| H_{I}' | e; 0 \rangle 
        \langle b; 0 | H_{I}' | g; \bv{k}\bv{\varepsilon} \rangle }
        { \hbar\omega - \hbar\omega_{0} + i\hbar (\Gamma/2 ) }
\end{equation} 
where $H_{I}'$ is the interaction Hamiltonian
\begin{equation}
    H_{I}' = -\bv{d} \cdot \bv{E}_{\perp} ( \bv{r} ) 
\end{equation}
and
\begin{equation}
    \bv{E}_{\perp}(\bv{r}) = i \sum_{j} 
        \left[ \frac{ \hbar \omega_{j} }{ 2\varepsilon_{0} L^{3}}  \right]^{1/2}
        \left( \hat{a}_{j}\bv{\varepsilon}_{j} e^{i\bv{k}_{j}\cdot\bv{r}} 
              - \hat{a}_{j}^{+}\bv{\varepsilon}_{j} e^{-i\bv{k}_{j}\cdot\bv{r}} 
        \right)
\end{equation}
Using the expressions for $H_{I}'$ and $\bv{E}_{\perp}(\bv{r})$ we obtain for the matrix element 
\begin{equation}
   \langle e; 0 | H_{I}' | g; \bv{k}\bv{\varepsilon} \rangle = 
       -i \sqrt{ \frac{ \hbar \omega }{2 \varepsilon_{0} L^{3} }} 
      \langle e | (\bv{d} \cdot \bv{\varepsilon} ) e^{-i\bv{k}\cdot\bv{r}}| g \rangle
\end{equation}
At this point the textbook treatment usually assumes that the atom is at the
origin and so the exponential inside the matrix element tipically does not show
up.  In our case the atom is in a lattice site and it occupies one of the
harmonic oscillator states of a lattice well.   The center of mass and internal
states of the atom can be separated,   and still using the labels $e$ and $g$
for the internal state of the atom, and writing the center of mass initial and final states as $| u \rangle$ and $|u'\rangle$ respectively we have
\begin{equation}
   \langle e; 0 | H_{I}' | g; \bv{k}\bv{\varepsilon} \rangle = 
       -i \sqrt{ \frac{ \hbar \omega }{2 \varepsilon_{0} L^{3} }} 
      \langle e | \bv{d} \cdot \bv{\varepsilon} | g \rangle 
      \langle v | e^{-i\bv{k}\cdot\bv{r}} | u \rangle
\end{equation}
and similarly
\begin{equation}
   \langle g; \bv{k}'\bv{\varepsilon}' | H_{I}' | e; 0\rangle = 
       i \sqrt{ \frac{ \hbar \omega' }{2 \varepsilon_{0} L^{3} }} 
      \langle g | \bv{d} \cdot \bv{\varepsilon}' | e \rangle 
      \langle u' | e^{i\bv{k}'\cdot\bv{r}} | v \rangle
\end{equation}
This gives for the matrix element
\begin{equation}
    \mathcal{T}_{fi} = \sum_{v} \frac{\sqrt{\omega \omega'}}{2\varepsilon_{0} L^{3}}
    \frac{ 
      \langle g | \bv{d} \cdot \bv{\varepsilon}' | e \rangle 
      \langle e | \bv{d} \cdot \bv{\varepsilon} | g \rangle 
      \langle u'| e^{i\bv{k}'\cdot\bv{r}} | v \rangle 
      \langle v | e^{-i\bv{k}\cdot\bv{r}} | u  \rangle
       }
        { \omega - \omega_{0} + i (\Gamma/2 ) }
\end{equation}
where we have summed over all possible intermediate center of mass states.  Note that the sum can be taken out using the closure relation $\sum_{v}|v\rangle\langle v| = \mathbb{1}$. 
%We note here that the phase factors $i\bv{k}\cdot\bv{r}$ that appear due to the center of mass motion of the atom are uncorrelated.   This comes out that way because the scattering of a photon is second order: first the photon gets absorbed and then it gets reemitted at a later time.   In neutron scattering for example, the scattering process is first order, represented by a contact interaction between the neutron and the nuclei in the crystal,  In this case the phase factor due to the center of mass motion shows up as
%\begin{equation}
%    \langle  e^{i (\bv{k}'-\bv{k}) \cdot \bv{r}} \rangle_{\mathrm{CM}}
%\end{equation}
%The norm squared of this term is the Debye-Waller factor.   It is necessary to
%consider if the uncorrelated expectation values that appear in the second order
%photon scattering are correct.  To this effect one needs to consider the time
%between absorption and emission processes which is on the order of $\Gamma$,
%the linewidth of the excited state.   This is much larger that the typical
%harmonic oscillator frequency in a lattice site, which is about 400 kHz for a
%lithium atom in a 50 recoil lattice.   With this in mind we may think about
%photon scattering as an effectively first order process (scattering and emission happen much
%quicker than the atom's center of mass can move)  and we may take the liberty
%of writing the transition matrix element which includes the center of mass
%motion as
%\begin{equation}
%\begin{split}
%   \langle \mathcal{T}_{fi} \rangle_{\mathrm{CM}} = &
%        \left \langle \frac{  
%        \langle g; \bv{k}'\bv{\varepsilon}'| H_{I}' | e; 0 \rangle 
%        \langle b; 0 | H_{I}' | g; \bv{k}\bv{\varepsilon} \rangle }
%        { \hbar\omega - \hbar\omega_{0} + i\hbar (\Gamma/2 ) }
%        \right \rangle _{\mathrm{CM}} \\
%     =& 
%    \frac{\sqrt{\omega \omega'}}{2\varepsilon_{0} L^{3}}
%    \frac{ 
%      \langle g | \bv{d} \cdot \bv{\varepsilon}' | e \rangle 
%      \langle e | \bv{d} \cdot \bv{\varepsilon} | g \rangle 
%      \langle  e^{i(\bv{k}'-\bv{k}) \cdot\bv{r}} \rangle_{\mathrm{CM}} }
%        { \omega - \omega_{0} + i (\Gamma/2 ) }
%\end{split}
%\end{equation} 
%where the more famliar Debye-Waller factor shows up.  

In our experiment we are driving a sigma-minus transition so we can consider only the projection of $\bv{d}$ onto $\bv{\varepsilon}_{-}$ 
\begin{equation}
     \langle e | \bv{d} \cdot \bv{\varepsilon} | g \rangle  \equiv
     d_{-} \bv{\varepsilon}_{-} 
\end{equation} 
which leads to 
\begin{equation}
    \mathcal{T}_{fi}  = 
    \frac{\sqrt{\omega \omega'}}{2\varepsilon_{0} L^{3}}
    \frac{ |d_{-}|^{2}  (\bv{\varepsilon}_{-}\cdot \bv{\varepsilon}' )
                       (\bv{\varepsilon}\cdot \bv{\varepsilon}_{-} )}
        { \omega - \omega_{0} + i (\Gamma/2 ) }
      \langle u' | e^{i(\bv{k}'-\bv{k}) \cdot\bv{r}} | u  \rangle
\end{equation}
We use the relation between $|d_{-}|^{2}$ and the linewidth of the transition
\begin{equation} 
    |d_{-}|^{2} =  3\pi \varepsilon_{0} \hbar
  \left( \frac{c}{\omega_{0}} \right)^{3}  \Gamma
\end{equation}
and also the approximation $\omega' \approx \omega \approx \omega_{0}$ (except
careful not to use this in the term in the denominator) to obtain
\begin{equation}
    \mathcal{T}_{fi} =
    \frac{ 3 } {k^{2}} 
    \frac{ \pi \hbar c } {  L^{3} } 
        (\bv{\varepsilon}_{-}\cdot \bv{\varepsilon}' )
                       (\bv{\varepsilon}\cdot \bv{\varepsilon}_{-} )
    \frac{ \Gamma/2  }
        { \omega - \omega_{0} + i (\Gamma/2 ) }
      \langle u' | e^{i(\bv{k}'-\bv{k}) \cdot\bv{r}} | u  \rangle
\end{equation}

The number of final states with energy between $\hbar c k'$ and $\hbar c ( k' + \mathrm{d}k')$  whose wave vector points inside the solid angle $\mathrm{d} \Omega'$ equals 
\begin{equation}
    \rho( \hbar c k') \hbar c \mathrm{d} k' \mathrm{d} \Omega ' = \frac{L^{3}}{8 \pi^{3} }  k'^{2} \mathrm{d} k' \mathrm{d} \Omega' 
\end{equation}
and we use sthis to sum over $k'$ and obtain the total transition rate 
\begin{equation}
\begin{split}
  \sum_{fu'} w_{fi} = & 
   \frac{2\pi}{\hbar}  \mathrm{d} \Omega' 
      \int_{0}^{\infty} \frac{k'^{2} \mathrm{d} k' }{ (2\pi / L^{3} ) ^{3} } 
   | \mathcal{T}_{fi} |^{2} 
   \delta( \hbar c k' - \hbar c k )  \\ 
   = & 
   \mathrm{d} \Omega' \frac{9}{4 k^{2}} \frac{ c } {L^{3} }
        |(\bv{\varepsilon}_{-}\cdot \bv{\varepsilon}' )
                       (\bv{\varepsilon}\cdot \bv{\varepsilon}_{-} ) |^{2}
    \left|
    \frac{ \Gamma/2  }
        { \omega - \omega_{0} + i (\Gamma/2 ) }  
      \langle u' | e^{i(\bv{k}'-\bv{k}) \cdot\bv{r}} | u  \rangle
     \right| ^{2} \\ 
   = & 
   \mathrm{d} \Omega' \frac{9}{4 k^{2}} \frac{ c } {L^{3} }
        |(\bv{\varepsilon}_{-}\cdot \bv{\varepsilon}' )
                       (\bv{\varepsilon}\cdot \bv{\varepsilon}_{-} ) |^{2}
    \frac{ (\Gamma/2)^{2}  }
        { \Delta^{2} +  (\Gamma/2 )^{2} }
    \left|
      \langle u' | e^{i(\bv{k}'-\bv{k}) \cdot\bv{r}} | u  \rangle
\right| ^{2} \\ 
\end{split} 
\end{equation}
If we consider the flux correspoding to the state of the initial photon $\phi = c/L^{3}$ then we can define the differential cross section 
\begin{equation}
 \frac{ \mathrm{d} \sigma } { \mathrm{d} \Omega'} =  
    \frac{\sum_{f} w_{fi} } { \mathrm{d} \Omega' \phi} = 
    \frac{9}{4 k^{2}} 
        |(\bv{\varepsilon}_{-}\cdot \bv{\varepsilon}' )
                       (\bv{\varepsilon}\cdot \bv{\varepsilon}_{-} ) |^{2}
    \frac{ (\Gamma/2)^{2}  }
        { \Delta^{2} +  (\Gamma/2 )^{2} }
    \left|
      \langle u' | e^{i(\bv{k}'-\bv{k}) \cdot\bv{r}} | u  \rangle
\right| ^{2}  
\end{equation}
From here we can write down the intensity at a detector located at $\bv{r}_{D}$ in the direction of $\mathrm{d} \Omega'$ as 
\begin{equation}
\begin{split}
I  =& \frac{1}{r_{D}^{2}} \frac{ \mathrm{d} \sigma } { \mathrm{d} \Omega'}
      I_{\mathrm{p}} 
   =   \frac{1}{r_{D}^{2}} \frac{ \mathrm{d} \sigma } { \mathrm{d} \Omega'}
      \frac{\hbar c k^{3}\Gamma}{6 \pi} \frac{I_{\mathrm{p}}}{I_{\mathrm{sat}}}  \\ 
   =& \frac{\hbar c k \Gamma}{r_{D}^{2}}  
    \frac{9}{4 (6\pi)} 
        |(\bv{\varepsilon}_{-}\cdot \bv{\varepsilon}' )
                       (\bv{\varepsilon}\cdot \bv{\varepsilon}_{-} ) |^{2}
    \left|
      \langle u' | e^{i(\bv{k}'-\bv{k}) \cdot\bv{r}} | u  \rangle
  \right| ^{2}
     \frac{ I_{\mathrm{probe}} / I_{\mathrm{sat}}}
        { 4(\Delta/\Gamma)^{2} + 1 }
\end{split}
\end{equation}
and if we identify the last term in the product as $\rho_{ee}$ (in the limit of low intensity) we can write down an expression for $\eta$ which was defined back in Eq.~(\ref{eq:Idef}), 
\begin{equation}
  \eta = \left[ \frac{\hbar c k \Gamma}{r_{D}^{2}}  
    \frac{9}{24\pi} \right]^{1/2} 
        (\bv{\varepsilon}_{-}\cdot \bv{\varepsilon}' )
                       (\bv{\varepsilon}\cdot \bv{\varepsilon}_{-} ) 
      \langle u' | e^{i(\bv{k}'-\bv{k}) \cdot\bv{r}} | u  \rangle
\end{equation}
Notice that we used the limit of low intensity to identify $\rho_{ee}$.  The factor $\eta$ which contains the angular part of the scattered photon distribution should not be affected if we use the expression for $\rho_{ee}$ that can be obtained from the optical Bloch equations, which is valid for any intensity, since the Bloch equations are not a perturbative treatment. 

\subsection{Summation for a collection of atoms} 

For a collection of atoms, the resulting fied is the sum of the field produced by each individual atom, so we have  
\begin{equation}
\begin{split}
\langle I (t) \rangle = & 
    \left\langle \left( \sum_{m} E_{m}^{(-)}(\bv{r}_{D}, t) \right)
            \left( \sum_{n} E_{n}^{(+)}(\bv{r}_{D}, t) \right) \right\rangle \\
\end{split} 
\end{equation}
where we have labeled the atoms with the indices $m$ and $n$.  Dropping the time dependence
\begin{equation}
\begin{split}
 I = &
    \sum_{mn}  \eta_{m}\eta_{n}^{*}  
              \left\langle S_{m+}S_{n-} \right\rangle
\end{split} 
\end{equation}
Using $S=\langle S \rangle + \delta S$, as we did above to obtain the coherent and icoherent parts of the intensity, we obtain
\begin{equation}
\begin{split}
\langle I \rangle = &
    \sum_{mn}  \eta_{m}\eta_{n}^{*} \left(
              \langle S_{m+}\rangle \langle S_{n-} \rangle  
            + \langle \delta S_{m+} \delta S_{n-} \rangle \right) \\
    = & 
    \sum_{mn}  \eta_{m}\eta_{n}^{*} 
        \langle  S_{m+}\rangle \langle S_{n-} \rangle 
   + \sum_{n} | \eta_{n}|^{2} \langle \delta S_{n+} \delta S_{n-} \rangle  
\end{split} 
\end{equation}
The steady state solutions of the optical Bloch equations will be used to evaluate the expectation values and we state them here:
\begin{gather} 
    \langle S_{\pm} \rangle =  u \pm i v  \\
    u =  \frac{ \Delta }{ \Gamma  \sqrt{ I_{\mathrm{p}} / I_{\mathrm{sat}}} } \frac{s}{ 1 + s } \\
    v =  \frac{ 1 } { 2 \sqrt{ I_{\mathrm{p}} / I_{\mathrm{sat}}} } \frac{s}{1+s} \\
\end{gather}
Putting this back in the equation for $I$  
\begin{multline}
 I = 
  \sum_{mn}  \eta_{m}\eta_{n}^{*}
    \left(
    \frac{ \Delta_{m} }{ \Gamma  \sqrt{ I_{\mathrm{p}} / I_{\mathrm{sat}}} } 
    \frac{s_{m}}{ 1 + s_{m} } 
   + i 
    \frac{ 1 } { 2 \sqrt{ I_{\mathrm{p}} / I_{\mathrm{sat}}} } \frac{s_{m}}{1+s_{m}} 
    \right) 
    \left(
    \frac{ \Delta_{n} }{ \Gamma  \sqrt{ I_{\mathrm{p}} / I_{\mathrm{sat}}} } 
    \frac{s_{n}}{ 1 + s_{n} } 
   - i 
    \frac{ 1 } { 2 \sqrt{ I_{\mathrm{p}} / I_{\mathrm{sat}}} } \frac{s_{n}}{1+s_{n}} 
    \right) \\
   + \sum_{n} | \eta_{n}|^{2} \frac{1}{2} \frac{ s_{n}^{2} } { (1 + s _{n} )^{2} } 
\end{multline}
\begin{multline}
 I  = 
  \sum_{mn}  \eta_{m}\eta_{n}^{*}
    \frac{ s_{m} s_{n} } { (I_{\mathrm{p}}/I_{\mathrm{sat}}) ( 1+s_{m} )( 1+s_{n} ) }
    \left(
        \frac{ \Delta_{m} \Delta_{n} }{ \Gamma^{2} } 
      + i \frac{ \Delta_{n} }{ 2 \Gamma } 
      - i \frac{ \Delta_{m} }{ 2 \Gamma } 
      + \frac{1}{4}  
    \right)  
   + \sum_{n} | \eta_{n}|^{2} \frac{1}{2} \frac{ s_{n}^{2} } { (1 + s _{n} )^{2} } 
\end{multline}
We proceed to split up the first sum into same-atom ($n=m$) and different atom ($n<m$) parts 
\begin{multline}
 I  = 
  \sum_{m<n} 
    \frac{ s_{m} s_{n} } { (I_{\mathrm{p}}/I_{\mathrm{sat}}) ( 1+s_{m} )( 1+s_{n} ) }
    \left(
        \eta_{m}\eta_{n}^{*}
    \left(
        \frac{ \Delta_{m} \Delta_{n} }{ \Gamma^{2} } 
      + i \frac{ \Delta_{n} }{ 2 \Gamma } 
      - i \frac{ \Delta_{m} }{ 2 \Gamma } 
      + \frac{1}{4}  
    \right)  \right. \\
   \left.  + 
        \eta_{n}\eta_{m}^{*}
    \left(
        \frac{ \Delta_{n} \Delta_{m} }{ \Gamma^{2} } 
      + i \frac{ \Delta_{m} }{ 2 \Gamma } 
      - i \frac{ \Delta_{n} }{ 2 \Gamma } 
      + \frac{1}{4}  
    \right) 
    \right)  \\
  + \sum_{n}  |\eta_{n}|^{2}
    \frac{ s_{n} s_{n} } { (I_{\mathrm{p}}/I_{\mathrm{sat}}) ( 1+s_{n} )( 1+s_{n} ) }
    \left(
        \frac{ \Delta_{n} \Delta_{n} }{ \Gamma^{2} } 
      + \frac{1}{4}  
    \right) \\ 
   + \sum_{n} | \eta_{n}|^{2} \frac{1}{2} \frac{ s_{n}^{2} } { (1 + s _{n} )^{2} } 
\end{multline}

\begin{multline}
 I  = 
  \sum_{m<n} 
    \frac{ s_{m} s_{n} } { (I/I_{\mathrm{sat}}) ( 1+s_{m} )( 1+s_{n} ) }
    2 \Re\left[ 
        \eta_{m}\eta_{n}^{*}
    \left(
        \frac{ \Delta_{m} \Delta_{n} }{ \Gamma^{2} } 
      + i \frac{ \Delta_{n} }{ 2 \Gamma } 
      - i \frac{ \Delta_{m} }{ 2 \Gamma } 
      + \frac{1}{4}  
    \right) \right] \\
  + \sum_{n}  |\eta_{n}|^{2}
    \frac{1}{2}	\frac{ s_{n} } { ( 1+s_{n} )^{2} }
   + \sum_{n} | \eta_{n}|^{2} \frac{1}{2} \frac{ s_{n}^{2} } { (1 + s _{n} )^{2} } 
\end{multline}

With this expression in hand we focus our attention on the terms $\eta_{m}\eta_{n}^{*}$ and $|\eta_{n}|^{2}$.  We start with the latter 
\begin{equation}
 |\eta_{n}|^{2} =  \frac{\hbar c k \Gamma}{r_{D}^{2}}  
    \frac{9}{24\pi} 
       | (\bv{\varepsilon}_{-}\cdot \bv{\varepsilon}' )
                       (\bv{\varepsilon}\cdot \bv{\varepsilon}_{-} ) |^{2}
      \langle u | e^{-i(\bv{k}'-\bv{k}) \cdot\bv{r}_{n}} | u'  \rangle
      \langle u' | e^{i(\bv{k}'-\bv{k}) \cdot\bv{r}_{n}} | u  \rangle
\end{equation}
and notice that we have to sum over output polarizations $\lambda$ and final center of mass states  $u'$, since our detector does not care about either. We obtain 
\begin{equation}
\begin{split}
 \sum_{\lambda u'}|\eta_{n}|^{2} = & \sum_{\lambda u'} \frac{\hbar c k \Gamma}{r_{D}^{2}}  
    \frac{9}{24\pi} 
       | (\bv{\varepsilon}_{-}\cdot \bv{\varepsilon}'_{\lambda} )
                       (\bv{\varepsilon}\cdot \bv{\varepsilon}_{-} ) |^{2}
      \langle u | e^{-i(\bv{k}'-\bv{k}) \cdot\bv{r}_{n}} | u'  \rangle
      \langle u' | e^{i(\bv{k}'-\bv{k}) \cdot\bv{r}_{n}} | u  \rangle \\
 = & \sum_{\lambda} \frac{\hbar c k \Gamma}{r_{D}^{2}}  
    \frac{9}{24\pi} 
       | (\bv{\varepsilon}_{-}\cdot \bv{\varepsilon}'_{\lambda} )
                       (\bv{\varepsilon}\cdot \bv{\varepsilon}_{-} ) |^{2}
      \langle u | e^{-i(\bv{k}'-\bv{k}) \cdot\bv{r}_{n}}  e^{i(\bv{k}'-\bv{k}) \cdot\bv{r}_{n}} | u  \rangle \\
 = & \sum_{\lambda} \frac{\hbar c k \Gamma}{r_{D}^{2}}  
    \frac{9}{24\pi} 
       | (\bv{\varepsilon}_{-}\cdot \bv{\varepsilon}'_{\lambda} )
                       (\bv{\varepsilon}\cdot \bv{\varepsilon}_{-} ) |^{2} \\
\end{split}
\end{equation}
where we have used the closure relation $\sum{u'}|u'\rangle\langle u'| = \mathbb{1}$.
Similarly for $\eta_{m}\eta_{n}^{*}$
\begin{equation}
\begin{split}
 \sum_{\lambda u'_{m} u'_{n}} \eta_{m}\eta_{n}^{*} = & 
 \sum_{\lambda u'_{m} u'_{n}} \frac{\hbar c k \Gamma}{r_{D}^{2}}  
    \frac{9}{24\pi} 
       | (\bv{\varepsilon}_{-}\cdot \bv{\varepsilon}'_{\lambda} )
                       (\bv{\varepsilon}\cdot \bv{\varepsilon}_{-} ) |^{2}
      \langle u_{n} | e^{-i(\bv{k}'-\bv{k}) \cdot\bv{r}_{n}} | u'_{n}  \rangle
      \langle u'_{m} | e^{i(\bv{k}'-\bv{k}) \cdot\bv{r}_{m}} | u_{m}  \rangle \\
\end{split}
\end{equation}
In this case we cannot use the closure relation because $n,m$ refer to
different atoms.   We simplify the treatment by considering only final states
for the atom that are the same as the initial state $u'=u$ (these are going to
have the largest matrix elements anyways), so the sum over $u'_{m},u'_{n}$ is
discarded.  We take the center of mass state of the atoms to be the
ground state of the single lattice site harmonic oscilator.  This leaves us
with 
\begin{equation}
\begin{split}
 \sum_{\lambda } \eta_{m}\eta_{n}^{*} = & 
 \sum_{\lambda } \frac{\hbar c k \Gamma}{r_{D}^{2}}  
    \frac{9}{24\pi} 
       | (\bv{\varepsilon}_{-}\cdot \bv{\varepsilon}'_{\lambda} )
                       (\bv{\varepsilon}\cdot \bv{\varepsilon}_{-} ) |^{2}
      \langle 0_{n} | e^{-i(\bv{k}'-\bv{k}) \cdot\bv{r}_{n}} | 0_{n}  \rangle
      \langle 0_{m} | e^{i(\bv{k}'-\bv{k}) \cdot\bv{r}_{m}} | 0_{m}  \rangle \\
\end{split}
\end{equation}

\subsubsection{Debye-Waller factor} 

For the center of mass expectation values we perform a translation $\bv{R}_{n}$ of the
coordinate system such that the position of the atom has a zero expectation
value $\langle \bv{r}_{n} \rangle = 0$.  A phase factor comes out that depends
on the position $\bv{R}_{n}$ of the lattice site in which the atom is located:
\begin{equation}
      \langle 0_{n} | e^{-i(\bv{k}'-\bv{k}) \cdot\bv{r}_{n}} | 0_{n}  \rangle 
    = e^{-i(\bv{k}'-\bv{k}) \cdot\bv{R}_{n}} 
      \langle 0_{n} | e^{-i(\bv{k}'-\bv{k}) \cdot\bv{r}_{n}} | 0_{n}  \rangle
\end{equation} 
We then use the equality $\langle e^{\hat{A}} \rangle = e^{\frac{1}{2} \langle \hat{A}^{2} \rangle }$, 
which is valid for a simple harmonic osscilator where $\hat{A}$ is any linear
combination of displacement and momentum operators of the oscillator.  This leaves us with
\begin{equation}
\begin{split}
      \langle 0_{n} | e^{-i(\bv{k}'-\bv{k}) \cdot\bv{r}_{n}} | 0_{n}  \rangle 
    = & e^{-i(\bv{k}'-\bv{k}) \cdot\bv{R}_{n}} 
      e^{ -\frac{1}{2} \left\langle [ (\bv{k}'-\bv{k}) \cdot\bv{r}_{n} ]^{2} \right\rangle } \\
    = & e^{ -i \bv{Q} \cdot \bv{R}_{n}} 
      e^{ -\frac{1}{2} \left\langle [ \bv{Q} \cdot\bv{r}_{n} ]^{2} \right\rangle } \\ 
    = & e^{ -i \bv{Q} \cdot \bv{R}_{n}}
      \prod_{i=x,y,z} e^{ - \frac{1}{2}Q_{i}^{2}\langle r_{ni} ^{2} \rangle } \\ 
    = & e^{ -i \bv{Q} \cdot \bv{R}_{n}}
      e^{-W} 
\end{split}
\end{equation} 
where we have defined the momentum transfer $\bv{Q} = \bv{k}' - \bv{k}$,  and the Debye-Waller factor $e^{-2W}$. 

Putting this back in the expression for $\eta_{m}\eta_{n}^{*}$ we get
\begin{equation}
\begin{split}
 \sum_{\lambda } \eta_{m}\eta_{n}^{*} = & 
 \sum_{\lambda } \frac{\hbar c k \Gamma}{r_{D}^{2}}  
    \frac{9}{24\pi} 
       | (\bv{\varepsilon}_{-}\cdot \bv{\varepsilon}'_{\lambda} )
                       (\bv{\varepsilon}\cdot \bv{\varepsilon}_{-} ) |^{2}
       e^{ i \bv{Q}( \bv{R}_{m} - \bv{R}_{n} ) } e^{-2W} 
\end{split}
\end{equation}
And if we now return to the expression for the intensity at the detector we have 
\begin{multline}
 I  = 
  \sum_{m<n} 
    \frac{ s_{m} s_{n} } { (I_{\mathrm{p}}/I_{\mathrm{sat}}) ( 1+s_{m} )( 1+s_{n} ) }
    2 \Re\left[ 
         \sum_{\lambda } \frac{\hbar c k \Gamma}{r_{D}^{2}}  
            \frac{9}{24\pi} 
               | (\bv{\varepsilon}_{-}\cdot \bv{\varepsilon}'_{\lambda} )
                               (\bv{\varepsilon}\cdot \bv{\varepsilon}_{-} ) |^{2}
               e^{ i \bv{Q}( \bv{R}_{m} - \bv{R}_{n} ) } e^{-2W} \right. \\ 
    \left.
    \left(
        \frac{ \Delta_{m} \Delta_{n} }{ \Gamma^{2} } 
      + i \frac{ \Delta_{n} }{ 2 \Gamma } 
      - i \frac{ \Delta_{m} }{ 2 \Gamma } 
      + \frac{1}{4}  
    \right) \right] \\ 
  + \sum_{n}  \frac{1}{2}
    \sum_{\lambda} \frac{\hbar c k \Gamma}{r_{D}^{2}}  
    \frac{9}{24\pi} 
       | (\bv{\varepsilon}_{-}\cdot \bv{\varepsilon}'_{\lambda} )
                       (\bv{\varepsilon}\cdot \bv{\varepsilon}_{-} ) |^{2} 
    \frac{ s_{n} } { 1 + s_{n} } 
\end{multline}

\begin{multline}
\label{eq:finalIdetector}
 I  =
 \left( 
 \frac{\hbar c k \Gamma}{r_{D}^{2}}  
     \frac{9}{24\pi} \Lambda 
  \right) \times \\
  \sum_{m<n} 
    \frac{ s_{m} s_{n} } { (I_{\mathrm{p}}/I_{\mathrm{sat}}) ( 1+s_{m} )( 1+s_{n} ) }
    2 \Re\left[ 
               e^{ i \bv{Q}( \bv{R}_{m} - \bv{R}_{n} ) } e^{-2W}  
    \left(
        \frac{ \Delta_{m} \Delta_{n} }{ \Gamma^{2} } 
      + i \frac{ \Delta_{n} }{ 2 \Gamma } 
      - i \frac{ \Delta_{m} }{ 2 \Gamma } 
      + \frac{1}{4}  
    \right) \right]  
  + \sum_{n}  \frac{1}{2}
    \frac{ s_{n} } { 1 + s_{n} } 
\end{multline}
where we have defined for brevity
\begin{equation}
 \Lambda = 
  \sum_{\lambda }
        | (\bv{\varepsilon}_{-}\cdot \bv{\varepsilon}'_{\lambda} )
                        (\bv{\varepsilon}\cdot \bv{\varepsilon}_{-} ) |^{2} 
\end{equation} 
It is good to see that for time-of-flight, where the Debye-Waller goes to zero due to large extent of the expanding atom wavefunctions,  this formula reduces to the the standard uncorrelated scattering for $N$ atoms with $\rho_{ee} = \frac{1}{2} \frac{s}{1+s}$.


\subsection{ Low intensity and large detuning limit}
We start from Eq.~(\ref{eq:finalIdetector}) and concentrate on the two sums, the first of which is 
\begin{equation} 
  \frac{  e^{-2W}}{2I_{\mathrm{p}}/I_{\mathrm{sat}}} \Re 
  \sum_{m<n} 
               e^{ i \bv{Q}( \bv{R}_{m} - \bv{R}_{n} ) } 
    \frac{ s_{m} s_{n} } {( 1+s_{m} )( 1+s_{n} ) }
    \left(
         4\Delta_{m} \Delta_{n} 
      + 2i \Delta_{n} 
      - 2i \Delta_{m}
      + 1
    \right)  
\end{equation}
where for simplicity we have now written the detunings in units of $\Gamma$.  We will split this up further into four terms 
\begin{align} 
  \frac{  e^{-2W}}{2I_{\mathrm{p}}/I_{\mathrm{sat}}} \Re \sum_{m<n} & 
      e^{ i \bv{Q}( \bv{R}_{m} - \bv{R}_{n} ) } 
      \frac{ s_{m} s_{n} } {( 1+s_{m} )( 1+s_{n} ) } 4 \Delta_{m} \Delta_{n} \\
  \frac{  e^{-2W}}{2I_{\mathrm{p}}/I_{\mathrm{sat}}} \Re \sum_{m<n} & 
      e^{ i \bv{Q}( \bv{R}_{m} - \bv{R}_{n} ) } 
      \frac{ s_{m} s_{n} } {( 1+s_{m} )( 1+s_{n} ) } 2 i \Delta_{n}  \\
  -\frac{  e^{-2W}}{2I_{\mathrm{p}}/I_{\mathrm{sat}}} \Re \sum_{m<n} & 
      e^{ i \bv{Q}( \bv{R}_{m} - \bv{R}_{n} ) } 
      \frac{ s_{m} s_{n} } {( 1+s_{m} )( 1+s_{n} ) } 2 i \Delta_{m}  \\
  \frac{  e^{-2W}}{2I_{\mathrm{p}}/I_{\mathrm{sat}}} \Re \sum_{m<n} & 
      e^{ i \bv{Q}( \bv{R}_{m} - \bv{R}_{n} ) } 
      \frac{ s_{m} s_{n} } {( 1+s_{m} )( 1+s_{n} ) }   
\end{align}

In the low intensity limit, and for a detuning such that $4\Delta_{m}^{2}, 4\Delta_{n}^{2} \gg 1 $ these four tend respectively to 
\begin{align} 
    e^{-2W} 2(I_{\mathrm{p}}/I_{\mathrm{sat}}) \Re \sum_{m<n} & 
      e^{ i \bv{Q}( \bv{R}_{m} - \bv{R}_{n} ) } 
      \frac{1}{ 4 \Delta_{m} \Delta_{n} }  \\
    e^{-2W} 2(I_{\mathrm{p}}/I_{\mathrm{sat}}) \Re \sum_{m<n} & 
      e^{ i \bv{Q}( \bv{R}_{m} - \bv{R}_{n} ) } 
      \frac{i}{ 8 \Delta_{n} \Delta_{m}^{2}} \\
  - e^{-2W} 2(I_{\mathrm{p}}/I_{\mathrm{sat}}) \Re \sum_{m<n} & 
      e^{ i \bv{Q}( \bv{R}_{m} - \bv{R}_{n} ) } 
      \frac{i}{ 8 \Delta_{n}^{2} \Delta_{m} } \\
    e^{-2W} 2(I_{\mathrm{p}}/I_{\mathrm{sat}}) \Re \sum_{m<n} & 
      e^{ i \bv{Q}( \bv{R}_{m} - \bv{R}_{n} ) } 
      \frac{1}{ 16 \Delta_{m}^{2} \Delta_{n}^{2} }
\end{align}
Furthermore, if we detune the light in between the two spin states then we can
use $\frac{1}{2\Delta_{m}} = \frac{1}{|\Delta|}S_{zm}$, where
$S_{zm}=\pm\frac{1}{2}$ is the spin state of the atom in site $m$, to obtain
\begin{align} 
    e^{-2W} \frac{2I_{\mathrm{p}}/I_{\mathrm{sat}}}{ |\Delta|^{2} }  \Re \sum_{m<n} & 
      e^{ i \bv{Q}( \bv{R}_{m} - \bv{R}_{n} ) } 
      S_{zm}S_{zn}  \\
    e^{-2W} \frac{2I_{\mathrm{p}}/I_{\mathrm{sat}}}{ 4 |\Delta|^{3} }  \Re \sum_{m<n} & 
      e^{ i \bv{Q}( \bv{R}_{m} - \bv{R}_{n} ) } 
       i S_{zn} \\
  - e^{-2W} \frac{2I_{\mathrm{p}}/I_{\mathrm{sat}}}{ 4 |\Delta|^{3} }  \Re \sum_{m<n} & 
      e^{ i \bv{Q}( \bv{R}_{m} - \bv{R}_{n} ) }
       i S_{zm} \\  
    e^{-2W} \frac{2I/I_{\mathrm{sat}}}{ 16 |\Delta|^{4} }  \Re \sum_{m<n} & 
      e^{ i \bv{Q}( \bv{R}_{m} - \bv{R}_{n} ) } 
\end{align}
We identify the first term and the last term as related to the spin structure factor and crystal structure factor that appear in Ted's
paper.   We have also two other terms show up here, which Ted discards in his
derivation, see more details on Sec.~\ref{sec:Ted}.    As was stated at the
beginning of this document, our main goal here is to connect the intensity
measured by our cameras with the spin-structure factor that is calculated by the
theorists.   In this last equation we see that the first term, the one related to the spin structure factor, is
going to have the main contribution to the intensity because it goes as $
|\Delta|^{-2} $, whereas the other terms go as larger powers of $1/|\Delta|$.  
If we neglect terms other than the spin structure factor then,  in the low intensity and large detuning limit (with the detuning set in between the two spin states), we obtain 
\begin{equation}
 I  =
 \left( 
 \frac{\hbar c k \Gamma}{r_{D}^{2}}  
     \frac{9}{24\pi} \Lambda 
  \right) \times 
  \left(
    e^{-2W} \frac{2I_{\mathrm{p}}/I_{\mathrm{sat}}}{ |\Delta|^{2} }  \Re \sum_{m<n}  
      e^{ i \bv{Q}( \bv{R}_{m} - \bv{R}_{n} ) } 
      S_{zm}S_{zn}  
  +  \frac{I_{\mathrm{p}}/I_{sat}}{ 4 \Delta^{2} } N
  \right)
\end{equation}
where we have made use of $s_{n}/(1+s_{n}) \approx (I_{\mathrm{p}}/I_{\mathrm{sat}})/ \Delta_{n}^{2}$ to carry out the sum over $n$ in Eq.~(\ref{eq:finalIdetector}).  We then manipulate the $n<m$ sum and the real part to obtain
\begin{equation}
 I  =
 \left( 
 \frac{\hbar c k \Gamma}{r_{D}^{2}}  
     \frac{9}{24\pi} \Lambda 
  \right) \times 
  \left(
    e^{-2W} \frac{I_{\mathrm{p}}/I_{\mathrm{sat}}}{ |\Delta|^{2} }  \sum_{m n}  
      e^{ i \bv{Q}( \bv{R}_{m} - \bv{R}_{n} ) } 
      S_{zm}S_{zn}  
  + (1- e^{-2W}) \frac{I_{\mathrm{p}}/I_{\mathrm{sat}}}{ 4|\Delta|^{2} }  N
  \right)
\end{equation}
In this formula the spin structure factor appears explicitly, we pull out some factors and get  
\begin{equation}
 I  =
 \frac{\hbar c k \Gamma}{r_{D}^{2}}  
     \frac{9}{24\pi} \Lambda
 \frac{I_{\mathrm{p}}/I_{\mathrm{sat}}}{ 4|\Delta|^{2} }  N
  \left(
   1 + 
    e^{-2W} \left( \frac{4}{ N }  \sum_{m n}  
      e^{ i \bv{Q}( \bv{R}_{m} - \bv{R}_{n} ) } 
      S_{zm}S_{zn}  - 1 \right )
     \right)
\end{equation}
After time-of-flight the Debye-Waller factor goes to zero due to the expanding size of the atomic wavefunctions and so  
\begin{equation}
 I_{\mathrm{TOF}} =
 \frac{\hbar c k \Gamma}{r_{D}^{2}}  
     \frac{9}{24\pi} \Lambda
 \frac{I_{\mathrm{p}}/I_{\mathrm{sat}}}{ 4|\Delta|^{2} }  N
\end{equation}
giving finally 
\begin{equation}
 \frac{I}{I_{\mathrm{TOF}}} = 1 +  e^{-2W}( S(\bv{Q}) - 1 ) 
\end{equation}
This has the expected form (we got this from David Huse), and it defines the spin structure factor as 
\begin{equation}
    S(\bv{Q}) =  \frac{4}{ N }  \sum_{m n}  
      e^{ i \bv{Q}( \bv{R}_{m} - \bv{R}_{n} ) } 
      S_{zm}S_{zn}  
\end{equation}

IMPORTANT REMARK:   We note here that this derivation which relates the
observed intensity to the spin structure factor relies on the saturation
parameter being much less than 1.   In our case we have
$I_{\mathrm{p}}/I_{\mathrm{sat}}\approx 25$ and $\Delta\approx 6.5$ which gives a saturation
parameter of $s=0.3$ which is less than 1, but maybe not entirely negligible.
In the near future we will attempt to use the exact expression for the
intensity at the detector which considers saturation effects to determine what
kind of corrections do we need to make to connect between our measurement and
the exact spin structure factor. 

\subsection{Walk through Ted's derivation to find missing terms}\label{sec:Ted}

We start with Ted's formula for the differential cross section
\begin{equation}
\begin{split}
\dsig{} =& \frac{9}{4k^{2}}
              \sum_{\lambda_{f}} | (\bv{e}_{\bv{k}_{f} \lambda_{f}}^{*} \cdot \bv{e}_{m} )
                                   (\bv{e}_{m}^{*} \cdot \bv{e}_{\bv{k}_{i} \lambda_{i}}^{*} )
                                 |^{2} \\
          & \times \sum_{ \sigma,\sigma', j, j' } [ \langle \hat{n}_{j\sigma}\hat{n}_{j'\sigma'}
              e^{ i \bv{K} \cdot ( \hat{\bv{r}}_{j} - \hat{\bv{r}}_{j'} ) } \rangle
              \bar{f}_{\sigma} {\bar{f}_{\sigma'}}^{*} ]
\end{split}
\end{equation}
and we abbreviate the sum over final polarizations as
\begin{equation}
 \Lambda = \sum_{\lambda_{f}} | (\bv{e}_{\bv{k}_{f} \lambda_{f}}^{*} \cdot \bv{e}_{m} )
                                   (\bv{e}_{m}^{*} \cdot \bv{e}_{\bv{k}_{i} \lambda_{i}}^{*} )
                                 |^{2}
\end{equation}
to obtain

\begin{equation}
\begin{split}
\dsig{} =& \frac{9\Lambda}{4k^{2}}
               \sum_{ \sigma,\sigma', j, j' } [ \langle \hat{n}_{j\sigma}\hat{n}_{j'\sigma'}
              e^{ i \bv{K} \cdot ( \hat{\bv{r}}_{j} - \hat{\bv{r}}_{j'} ) } \rangle
              \bar{f}_{\sigma} {\bar{f}_{\sigma'}}^{*} ] \\
\end{split}
\end{equation}


We will begin by disecting the sum that appears in the elastic cross section. The thermal average factorizes and
\begin{equation}
\begin{split}
\langle \hat{n}_{j\sigma}\hat{n}_{j'\sigma'} \rangle = &\, \langle (\frac{1}{2} + \sigma\hat{S}_{zj} )( \frac{1}{2} + \sigma'\hat{S}_{j'} ) \rangle \\
       = &\, \frac{1}{4} + \frac{1}{2} \langle \sigma \hat{S}_{zj} \rangle +
\frac{1}{2} \langle \sigma'\hat{S}_{zj'} \rangle + \langle \sigma\sigma'\hat{S}_{zj}\hat{S}_{zj'}
\rangle \\
\end{split}
\end{equation}
For this last step
\begin{center}
  \begin{tabular}{ c }
    $\hat{n}_{i\uparrow} + \hat{n}_{i\downarrow} = 1$ \\
    $\sigma = \pm 1$ \\
    $\hat{S}_{zi} = \frac{1}{2}( \hat{n}_{i\uparrow} - \hat{n}_{i\downarrow} )$
  \end{tabular}
\end{center}
We can manually perform the sums over $\sigma\sigma'$ for each of this four terms and define $\alpha$, $\beta$, and $\kappa$,
\begin{equation}
\begin{split}
\frac{1}{4} \sum_{\sigma\sigma'} \bar{f}_{\sigma} \bar{f}_{\sigma'}^{*} = &\,
            \frac{1}{4}(\bar{f}_{\uparrow} + \bar{f}_{\downarrow})( \bar{f}_{\uparrow}^{*} + \bar{f}_{\downarrow}^{*} ) \\
       =&\, \frac{1}{4} | \bar{f}_{\uparrow} + \bar{f}_{\downarrow} | ^{2} \equiv \alpha
\end{split}
\end{equation}

\begin{equation}
\begin{split}
\frac{1}{2} \sum_{\sigma\sigma'} \sigma'\bar{f}_{\sigma} \bar{f}_{\sigma'}^{*} = &\,
  \frac{1}{2} (
            - \bar{f}_{\downarrow}\bar{f}_{\downarrow}^{*}
            + \bar{f}_{\downarrow}\bar{f}_{\uparrow}^{*}
            - \bar{f}_{\uparrow}\bar{f}_{\downarrow}^{*}
            + \bar{f}_{\uparrow}\bar{f}_{\uparrow}^{*} ) \equiv \kappa
\end{split}
\end{equation}

\begin{equation}
\begin{split}
\frac{1}{2} \sum_{\sigma\sigma'} \sigma \bar{f}_{\sigma} \bar{f}_{\sigma'}^{*} = &\,
  \frac{1}{2} (
            - \bar{f}_{\downarrow}\bar{f}_{\downarrow}^{*}
            - \bar{f}_{\downarrow}\bar{f}_{\uparrow}^{*}
            + \bar{f}_{\uparrow}\bar{f}_{\downarrow}^{*}
            + \bar{f}_{\uparrow}\bar{f}_{\uparrow}^{*} ) \equiv \kappa^{*}
\end{split}
\end{equation}

\begin{equation}
\begin{split}
\sum_{\sigma\sigma'} \sigma\sigma' \bar{f}_{\sigma} \bar{f}_{\sigma'}^{*} = &\,
            (\bar{f}_{\uparrow} + \bar{f}_{\downarrow})( \bar{f}_{\uparrow}^{*} + \bar{f}_{\downarrow}^{*} ) \\
       =&\, | \bar{f}_{\uparrow} - \bar{f}_{\downarrow} | ^{2} \equiv \beta
\end{split}
\end{equation}

The elastic cross section is then
\begin{equation}
\begin{split}
\dsig{E} =& \frac{9\Lambda}{4k^{2}}
               \sum_{ j j' } \langle
              e^{ i \bv{K} \cdot ( \hat{\bv{r}}_{j} - \hat{\bv{r}}_{j'} ) } \rangle
             \left( \alpha + \langle \hat{S}_{zj}\rangle \kappa + \langle \hat{S}_{zj'} \rangle \kappa^{*}
                     + \langle \hat{S}_{zj} \hat{S}_{zj'} \rangle \beta \right)
\end{split}
\end{equation}

In the Bragg scattering paper by Ted, the two central terms are ignored, but there is no mention of why they are ignored.  As we saw above they also appear in the treatement of scattering that we have undertaken here.  


\section{Numerical calculations}

For the numerical calculation of the scattered intensity we will consider a
lattice with $L\times L\times L$ sites in which there is a core of size
$L_{\mathrm{AFM}} \times L_{\mathrm{AFM}} \times L_{\mathrm{AFM}}$ in which the
atoms have antiferromagnetically ordered spins.   The distribution of the spins
outside the core is random, but the spin imbalance is constrained to be zero,
that is there is an equal number of atoms in state $|1\rangle$ and state
$|2\rangle $  occupying the $L^{3}$ sites in the lattice. 

For the numerical calculation we start from Eq.~(\ref{eq:finalIdetector}) and replace the sum over $m<n$ with an unrestricted sum over $m,n$.  This removes the real part, and we have to subract again some $m=n$ terms, the result is 
\begin{multline}
 I  =
 \left( 
 \frac{\hbar c k \Gamma}{r_{D}^{2}}  
     \frac{9}{24\pi} \Lambda 
  \right) \times \\
  \frac{ e^{-2W} }{ 4 (I_{\mathrm{p}}/I_{\mathrm{sat}})} \sum_{mn} 
    \frac{ s_{m} s_{n} } { ( 1+s_{m} )( 1+s_{n} ) }
               e^{ i \bv{Q}( \bv{R}_{m} - \bv{R}_{n} ) }
    \left(
        4 \Delta_{m} \Delta_{n}
      + 2 i \Delta_{n} 
      - 2 i \Delta_{m}
      + 1
    \right)  
  + \sum_{n}  \frac{1}{2}
    \frac{ s_{n} } { 1 + s_{n} } \left( 1 - \frac{e^{-2W}}{1+s_{n}} \right)
\end{multline}
To facilitate the numerical calculation we will split this up as 
\begin{multline}
 I  =
 \left( 
 \frac{\hbar c k \Gamma}{r_{D}^{2}}  
     \frac{9}{24\pi} \Lambda 
  \right) \times \\
  \left[
  \frac{ e^{-2W} }{ 4 (I_{\mathrm{p}}/I_{\mathrm{sat}})}
  \left( 
    2 \sum_{m}  
    \frac{ s_{m} }
         {  1 + s_{m} } \Delta_{m} e^{i\bv{Q}\cdot\bv{R}_{m} }
    2 \sum_{n} 
    \frac{ s_{n} }
         {  1 + s_{n} } \Delta_{n} e^{-i\bv{Q}\cdot\bv{R}_{n} }  \right. \right.\\
    + 2i\sum_{m}  
    \frac{ s_{m} }
         {  1 + s_{m} } e^{i\bv{Q}\cdot\bv{R}_{m} }
    \sum_{n} 
    \frac{ s_{n} }
         {  1 + s_{n} } \Delta_{n} e^{-i\bv{Q}\cdot\bv{R}_{n} }   \\
    - 2i\sum_{m}  
    \frac{ s_{m} }
         {  1 + s_{m} } \Delta_{m} e^{i\bv{Q}\cdot\bv{R}_{m} }
    \sum_{n} 
    \frac{ s_{n} }
         {  1 + s_{n} } e^{-i\bv{Q}\cdot\bv{R}_{n} }   \\
    \left.
    +\sum_{m}  
    \frac{ s_{m} }
         {  1 + s_{m} } e^{i\bv{Q}\cdot\bv{R}_{m} }
    \sum_{n} 
    \frac{ s_{n} }
         {  1 + s_{n} } e^{-i\bv{Q}\cdot\bv{R}_{n} }  \right) \\
   \left.
  + \sum_{n}  \frac{1}{2}
    \frac{ s_{n} } { 1 + s_{n} } \left( 1 - \frac{e^{-2W}}{1+s_{n}} \right) \right]
\end{multline}
The following sums appear and we define some shorthand notation for them
\begin{align} 
     \Phi \equiv & 
     \sum_{m}  
    \frac{ s_{m} }
         {  1 + s_{m} } \Delta_{m} e^{i\bv{Q}\cdot\bv{R}_{m} } \\
     \Upsilon \equiv &
     \sum_{m}  
    \frac{ s_{m} }
         {  1 + s_{m} } e^{i\bv{Q}\cdot\bv{R}_{m} } \\
     S \equiv & 
     \sum_{n}  
     \frac{ s_{n} } { 1 + s_{n} }  \\
     V \equiv & 
     \sum_{n}  
     \frac{ s_{n} } { (1 + s_{n})^{2}}  \\
\end{align}
We can then write the intensity as 
\begin{equation}
\begin{split}
 I  = &
 \left( 
 \frac{\hbar c k \Gamma}{r_{D}^{2}}  
     \frac{9}{24\pi} \Lambda 
  \right)
  \left[
  \frac{ e^{-2W} }{ 4 (I_{\mathrm{p}}/I_{\mathrm{sat}})}
  \left( 
    4 \Phi \Phi^{*}
    + 2i \Upsilon \Phi^{*} 
    - 2i \Phi \Upsilon^{*}
    + \Upsilon \Upsilon^{*}  
          \right)
  + \frac{1}{2}S - \frac{ e^{-2W}}{2} V 
\right] \\ 
\end{split}
\end{equation}
For the numerical evaluation we will use $\frac{\hbar c k \Gamma}{r_{D}^{2}}  
     \frac{9}{48\pi}$  as a unit for the intensity, so we can finally simplify the expression to
\begin{equation}
\begin{split}
  I  = &
 \Lambda 
  \left[
  S + 
  e^{-2W} \left(  
  \frac{ | \Upsilon - 2 i \Phi |^{2} }{  2 (I_{\mathrm{p}}/I_{\mathrm{sat}})}
   - V \right)
  \right] \\ 
\end{split}
\end{equation}

 
%\subsection{ Analytical dependence of crystal and magnetic structure factors on \bv{Q}}
%
%\subsubsection{Crystal}
%
%\begin{equation}
%C(\bv{Q}) = \sum_{jk} e^{i \bv{Q} \cdot (\bv{R}_{j} - \bv{R}_{k} )} 
%\end{equation}
%
%We can make the substitution $\bv{r}_{k} = \bv{R}_{j} -
%\bv{R}_{k}$.  For an infinite crystal, the sum over all pairs $jk$ may be replaced by the sum over all
%sites $j$ and then sum over all $\bv{r}_{k}$.
% 
%\begin{equation}
%C(\bv{Q}) = N\sum_{k} e^{i \bv{Q} \cdot \bv{r}_{k}} 
%\end{equation}
%
%\subsubsection{Magnetic}
%
%\begin{equation}
%S(\bv{Q}) = \sum_{jk} e^{i \bv{Q} \cdot (\bv{R}_{j} - \bv{R}_{k} )} \langle S_{zj} S_{zk} \rangle
%\end{equation}
%
%In the zero temperature AFM state there is a staggered magnetization, such that 
%
%\begin{equation}
%\langle S_{zj}S_{zk} \rangle = e^{i \bv{q} \cdot (\bv{R}_{j} - \bv{R}_{k} )} 
%\ \ \ \mathrm{where} \ \ \  
%\bv{q} = \frac{2\pi}{a} \left( \frac{1}{2}\, \frac{1}{2}\, \frac{1}{2} \right)
%\end{equation}
%
%At finite temperature, the staggered magnetization will have a finite correlation length \Lc, which results in 
%
%\begin{equation}
%\langle S_{zj}S_{zk} \rangle = e^{i \bv{q} \cdot (\bv{R}_{j} - \bv{R}_{k} )} e^{-|\bv{R}_{j} -\bv{R}_{k} |/ \Lc }
% \ \ \ \mathrm{where} \ \ \   
%\bv{q} = \frac{2\pi}{a} \left( \frac{1}{2}\, \frac{1}{2}\, \frac{1}{2} \right)
%\end{equation}
%
%\begin{equation}
%S(\bv{Q}) = \sum_{jk} e^{i (\bv{Q} + \bv{q} )\cdot (\bv{R}_{j} - \bv{R}_{k} )} e^{-|\bv{R}_{j} -\bv{R}_{k} |/ \Lc }
%\end{equation}
%
%At this point we can make the substitution $\bv{r}_{k} = \bv{R}_{j} -
%\bv{R}_{k}$.  For an infinite crystal, the sum over all pairs $jk$ may be replaced by the sum over all
%sites $j$ and then sum over all $\bv{r}_{k}$. 
%
%\begin{equation}
%S(\bv{Q}) = \sum_{jk} e^{i (\bv{Q} + \bv{q} )\cdot \bv{r}_{k} } e^{-r_{k}/ \Lc } = N \sum_{k} e^{i (\bv{Q} + \bv{q} )\cdot \bv{r}_{k} } e^{-r_{k}/ \Lc }
%\end{equation}

\bibliographystyle{osa}
\bibliography{bragg}

\end{document}




